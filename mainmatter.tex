% !TeX encoding = UTF-8
% !TeX root = MAIN.tex

\chapter{Introduction}
\label{ch:introduction}

The software company Dynatrace\footnote{\url{https://www.dynatrace.com/}} is offers two products called OneAgent and ActiveGate.
Both products are also available in a containerized format.
Using this containerized format, it is possible to run them in computer clusters managed by software such as Kubernetes\footnote{\url{https://kubernetes.io/}}.

However, in order to run correctly, applying some cluster specific configuration is necessary.
These configurations are mostly provided using files in the YAML format\cite{UnderstandingKubernetesObjects}.
Before applying these configurations, they have to be created and edited by a customer to fit their environment.

In order to remove the necessity of complicated configurations and possibility of human error, Dynatrace maintains the Dynatrace Operator (DTO).
The DTO is an open-source project, which can be deployed on a computer cluster.
After deployment, it allows applying a special configuration file, which contains all needed configuration options a customer must or can apply to either a OneAgent or ActiveGate deployment.
It then automatically manages the deployment of these products and their lifecycle.

This operator\cite{OperatorPattern} is released, in the form of deployment files to a variety of platforms.
These platforms, at the time of writing, are
its GitHub\footnote{\url{https://github.com/}} repository\footnote{\url{https://github.com/Dynatrace/dynatrace-operator}},
the RedHat Container Catalog\footnote{\url{https://catalog.redhat.com/}},
the Community Operators repository\footnote{\url{https://github.com/operator-framework/community-operators}},
the ArtifactHub\footnote{\url{https://artifacthub.io/}} of Helm\footnote{\url{https://helm.sh/}},
the Rancher\footnote{\url{https://rancher.com/}} repository\footnote{\url{https://github.com/rancher/helm3-charts}} for Helm charts and
the Google Kubernetes Engine\footnote{\url{https://cloud.google.com/kubernetes-engine}} (GKE) Marketplace\footnote{\url{https://cloud.google.com/marketplace}}.

The releases to all of these platforms are currently done manually.
Since every platform has its own special handling of a release, every DTO release takes a lot of time.
The work time needed for a single release is between four workdays and one and a half workweeks.
These times can be seen in figure X.

% TODO: Insert figure of tickets here.

