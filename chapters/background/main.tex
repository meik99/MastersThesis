\chapter{Background}\label{ch:background}

This thesis and its system mostly concerns itself with the topic of DevOps.
What DevOps means and why it is relevant is explained further in this chapter.
More specifically, the subtopics of automation, CI/CD pipelines and testing are looked at.
Concerning the topic of testing, variability-testing and smoke-testing are most relevant.

\section{DevOps}\label{sec:devops}

The term DevOps often refers to the stages of a product that are between development and release.
Ebert et al.\ describe it as a bridge between "the two worlds of development and operations"\cite{DevOps}.
However, they describe continuous integration and continuous delivery as tools used by DevOps practices rather than being such practices themselves.
They describe it as tools which allow developers to release and deploy a piece of software in a short timeframe.

Leite et al., on the other side, describes DevOps as a set of practices to release software in a short timeframe\cite{ASurveyofDevOpsConceptsandChallenges}.
Furthermore, Chen Lianping describes continuous delivery as an approach rather than a tool\cite{ContinuousDeliveryHugeBenefitsButChallengesToo}.
Apparently, different definitions for the term ''DevOps'' as well as ''Continuous Delivery'' exist.
Therefore, and for the purposes of this paper, the term DevOps is meant as follows.

A set of practices, which do not influence the development of software systems directly.
It does, however, automate large or all parts of the deployment and delivery process.
Since a system to deploy and deliver is needed for a deployment or delivery process, it follows that it is dependent on the development process.
Furthermore, stable and well functioning software is, trivially, more desirable than unstable or malfunctioning software.
It follows that the automation of a delivery process must include ways to ensure, that delivered software is as stable and well functioning as possible.
Otherwise, manual testing and delivery would be needed to ensure this state, which would ultimately defeat the purpose of said process.
In summary, DevOps describes the process of automatically testing and delivering a software system.

From this definition two parts become apparent, automatic testing and automatic delivery.
These parts correspond to the practices of continuous integration and continuous delivery.

