\section{Kubernetes}\label{sec:kubernetes}

A large part of this thesis concerns itself with a technology called \textbf{Kubernetes}.
Kubernetes is a complex software system and care must be taken to establish different parts and concepts of it.
This section covers the basic information needed to understand how it works and what certain terminology means in its context.

First, \textbf{Images} and \textbf{Containers}~\cite{docker-image,kubernetes-images,kubernetes-containers} act as the base of Kubernetes.
An image can be seen as a template to a container.
It includes instructions to be run to create a container.
These instructions can range from simple commands to complex, multistep build instructions.
Any image can also serve as the starting point for a new image.
This allows the creation of complex systems that can be instantiated as containers at any time.

A Kubernetes system is structured as a \textbf{cluster} and therefore called a ``Kubernetes cluster''~\cite{kubernetes-cluster}.
It is called a cluster because it is made of a cluster of \textbf{node}s.
In the context of Kubernetes, a node is a computer that serves as a part of a cluster.

The smallest clusters typically have three nodes~\cite{kubernetes-cluster}.
One control node and multiple worker nodes.
Although it is possible to have clusters with only one or two nodes, but they are mostly only used for development and debugging purposes~\cite{minikube}.
Bigger clusters and clusters that are made for software systems with high availability requirements can also have multiple control nodes for redundancy.
In such a cluster, worker nodes interact with one control node at a time, with other control nodes taking over if one fails.

\pagebreak

\textbf{Pods}~\cite{kubernetes-pods} are the smallest units of work a Kubernetes cluster uses to handle workloads.
A pod can consist of multiples containers, grouping them logically.
When deploying a container that way, a pod can manage the environment of the pod, such as mounted directories, network capabilities or environment variables.
Furthermore, a pod can have three different kinds of containers, normal containers, \textbf{init-containers} and ephemeral containers.
Normal containers describe the main workload of the pod.
Usually the main application is run inside one container, while other tasks, such as logging, is done in another.
This pattern is called Sidecar pattern and these pods are commonly referred to as \textbf{sidecar containers}~\cite{sidecar-container}.
These containers are only started, however, after all of a pods init-containers have succeeded.
For example, a pod may have a web server as a container that it runs continuously, but has an init container that executes setup tasks that need to run before the server starts.
Finally, ephemeral containers are containers added to an already deployed pod to be used for debugging.

\textbf{ReplicaSets}~\cite{kubernetes-replicaset} are the simplest form of deploying multiple instances of pods.
They are defined by providing a template of a pod and the number of replicas of the given template.
The replicaset then does its best to deploy the specified amount of pods on the cluster.
It schedules new ones if there are too few and deletes pods if there are too many.
However, sometimes it is not possible for a specific amount of pods to actually be deployed due to resource constraints of the cluster and its nodes.

\textbf{Deployments}~\cite{kubernetes-deployments} build on top of replica sets.
They are also defined by a pod specification and the required amount of replicas.
However, a deployment does not manage the pods, but uses a replica set to do so.
This enables a deployment to, for example, do rolling updates automagically\footnote{Automatically, but being reminiscent of wizardry}.
When a deployment is changed, it can decrease the desired amount of replicas of the original replicaset, while gradually increasing it for the new one.
An application can be updated seemingly and gradually to a new version, to avoid service outages in case of bugs or server downtimes.

Similar to deployments, \textbf{StatefulSets}~\cite{kubernetes-statefulsets} handle the deployment of pods.
The difference is, that the name, i.e., the identity, of a pod is guaranteed.
When a pod is created through a replica set or a deployment, the pod is assigned a semi-random name consisting of the deployment or replicaset name and random string.

A pod created by a stateful set on the other hand is usually named with the name of the stateful set with the index of the pod at the end.
For example, a pod of a deployment with the name of ``my-app'' results in the pod being named ``my-app-asd123-fgh456''.
A pod of a stateful set of the same name would result in the name being ``my-app-1''.
Furthermore, the naming is made consistent.
If a stateful set controls three pods, ``pod-1'', ``pod-2'', and ``pod-3'', and ``pod-2'' is deleted, it will try to deploy a new pod with the name ``pod-2''.

The \textbf{Kubernetes API}~\cite{kubernetes-api} is the part of Kubernetes that receives resource requests.
I.e., if an administrator wants to create resource, such as a pod, a HTTP request is sent to the Kubernetes API describing that resource.
From there, it is then applied to the cluster and other Kubernetes components are notified.
Interacting with this API is commonly abstracted to the Kubernetes CLI for manual, or client libraries for automatic communication.

\textbf{Webhooks}~\cite{kubernetes-webhooks} are used to catch requests for pods and modify them before they are scheduled.
Whenever a resource is modified in a cluster, it is handled as a HTTP request.
These HTTP requests can then be sent to a webhook, before they are returned to Kubernetes for further processing.
Webhooks that are used to mutate a resource are called mutating admission webhooks.
A different type of webhook also exists called validating admission webhook.
This type of webhook does not mutate a resource but is used to either allow or deny a resource being modified.

\textbf{Manifests}~\cite{kubernetes-manifests} represent the state of a Kubernetes cluster as a configuration file.
These configuration files are written in YAML.
When such a configuration is applied to a cluster, Kubernetes tries its best to create the state that is described by this configuration.
Multiple resource can be defined in a single configuration.
Multiple configurations can be applied to a single cluster.
Note well that these files are static, meaning that they do not accept the application of variables when applied to Kubernetes.

In order to manipulate manifests dynamically \textbf{Helm charts}~\cite{helm-charts} can be used.
A Helm chart is a package containing templates of Kubernetes resources and metadata.
Helm charts also include a way to configure these templates.
When a Helm chart is applied to a Kubernetes cluster, this configuration can be changed, either manually or dynamically.
The resources are then generated based on the supplied configuration.
Helm charts can also be compiled to an archive and distributed as such or as part of a repository.

\pagebreak

\textbf{Operators}~\cite{kubernetes-operator} are custom systems that interact with the Kubernetes API from within a Kubernetes cluster.
To do so, they often supply their own custom resource with which an administrator can configure an operator.
An operator's functionality is custom, but it is commonly deploying other resources on demand or maintaining them.
As previously mentioned, Kubernetes manifests are usually static, but an operator can be used to configure these resources on the line.

The \textbf{Operator Framework}~\cite{operator-framework, operator-lifecycle-manager} is a set of libraries which makes it easier to create operators in the Golang language.
In combination with the \textbf{Operator Lifecycle Manager (OLM)}~\cite{operator-lifecycle-manager}, operators created with this framework can be more easily run in a range of clusters.
OLM describes and handles how an operator can be installed and managed on a cluster.
Providers that offer Kubernetes clusters as a service can include support for OLM.
These providers than commonly offer a marketplace from which an operator can be installed.

\textbf{Amazon Web Services (AWS) and GCS (Google Cloud Service)}~\cite{eks,gke} are two of the many providers of Kubernetes cluster that exists.
These two are important for this thesis, as mainly their infrastructure is used.
The specific products that are used to create a cluster are called Elastic Kubernetes Service (EKS) and Google Kubernetes Engine (GKE) respectively.
EKS is later used to create clusters that run Openshift.

\textbf{Openshift}~\cite{openshift} is a software system built on top of a Kubernetes cluster.
It abstracts different aspects of cluster administration, such as resource or credential management.
Furthermore, it provides a marketplace from RedHat that uses OLM to install and manage operators.
