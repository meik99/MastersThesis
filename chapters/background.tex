
\chapter{Background}\label{ch:background}

In order to answer \textit{Q1}, a system is developed to help abstract the complexity of maintaining higher level tests.
This system's purpose is to use a template of a configuration file and then generate a concrete configuration for a pipeline.
Since the pipeline configuration is stored in code, as discussed before, the template must contain a common basis of this code.

This method has a few advantages to manually maintaining configurations.
First, if a pipeline configuration contains a fault, a fix in the template fixes all concrete configurations derived from it.
Secondly, only one configuration must be maintained, avoiding a problem analogous to one known as ''shotgun-surgery''\footnote{\url{https://refactoring.guru/smells/shotgun-surgery}} in software engineering, were one small changes must be made in many different files.
Thirdly, once the template is written, the domain knowledge is sufficiently abstracted for other team members.
Using the system to generate a concrete configuration does not necessitate knowledge about the domain of pipeline configurations.

As previously discussed, on of the main difficulties of testing highly scaled and complex systems, is the need of setting up complex environments.
Later it is shown, that for certain technologies, the setup for a single test may take multiple hours.
For large test suits, it is therefore unpractical to run all tests everytime, as the time-cost is too high.
In order to mitigate this issue a technique called similarity-based variability testing is used.

Al-Hajjaji et. al., describe this technique in the context of product line testing\cite{SimilarityBasedPrioritizationInSoftwareProductLineTesting}.
A software product line is described as a base software system, which is extended with features that make use of the base system.
Due to the high variability, not all combinations and features can be practically tested.
Therefore, an algorithm is proposed, that selects tests based on similarity.

This algorithm uses a set of possible configurations, \textit{C} and a predefined definition of their similarity \textit{S} to each other\footnote{The used variable names C and S do not correspond to the variable names in the original algorithm.}.
For example, given mobile phones, two phones that both integrate a GPS transceiver are more similar to each other, than one phone that does and one that does not.
It then takes the first configuration from the input set and adds it to the result.
Then, from the selected configuration, the next, least similar, configuration is selected.
This selection is added to the result and the next, least similar, configuration to this second one is chosen.
The algorithm then continues until all configurations are sorted in the result or a limit is reached.

This algorithm, while used in a different context, is also valuable when testing complex system.
As previous discussed, for sufficiently complex environments and large test suits, not all tests can realistically be run at all times.
Analogous to configurations, metadata can also be added to define the similarity of tests to each other.
A similar algorithm as described above can then be used to sort and select tests which are possible to run in a limited timeframe.
In section \ref{sec:dynamic-testing-framework} it is shown that this algorithm helps in implementing a solution for \textit{Q1}. %%TODO: replace "later section" with actual section

In order to answer \textit{Q2}, another pipeline is created for a specific system.
This pipeline contains steps to gather information, build necessary files and then deploy those files to different platforms.
The platforms to be deployed are specific marketplaces for this system and will be discussed in section \ref{sec:release-pipeline}

The goal is to find and use data, that is common to each platform and prepare it as much as possible before distributing it.
Each platform may have special steps that need to be taken into account.
Combining these steps in small scripts may increase the maintainability of the whole pipeline.
