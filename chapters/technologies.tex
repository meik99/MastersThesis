\chapter{Technologies and concepts}\label{ch:technologies-used}

An overview of the technologies that have been used is given in this section.
The name and, if applicable, the acronym of a technology is stated followed by a short description of it.
Then follows the reason for its involvement in this project.

\textbf{Concourse} is a CI/CD system that can be used to automate certain tasks and group them into a pipeline.
Tasks can then be run either concurrently or sequentially in the context of the pipeline they are grouped in.
In addition, resources to be used can be defined and added to tasks, so they are available to them.
Such resources can be files, online resources, storage locations or any number of information a task may need.
Concourse is used to implement the pipeline answering \textit{Q2}.

\textbf{YAML and JSON} are data formats.
They define the structure of information that makes it easier to transform and read the information in an automated manner.

\textbf{Golang (Go)} is a programming language.
It is the language of choice that is used to implement a tool to answer \textit{Q1}.

\textbf{BitBucket} is a system that utilizes \textbf{Git} to store project files.
Git is used to version a project and make it easier for a team of people to simultaneously work on a project.
A project that is worked on in this manner is commonly called a repository.
BitBucket itself then stores the repository online to ease access to it.

\textbf{OneAgents and ActiveGates} are Dynatrace proprietary software systems.
The Dynatrace OneAgent\cite{oneagents} is a software system to monitor other systems and collect metrics of them.
These metrics can range from memory or cpu usage to more complex ones such as network connections or amount of specific function calls.
Two different kinds of OneAgents exist, a host agent and a special agents.
The host agents purpose is to monitor the host of any given system, for example, a server on which a web application is deployed on.
Then, for different technologies, different special agents exist.
A special agent's purpose is to monitor an application, such as the web application mentioned earlier.

An ActiveGates\cite{activegates} main purpose is to ac as a proxy between OneAgents and the Dynatrace cluster.
It can be used to buffer information sent by a OneAgent inside a local network, before uploading it to the main Dynatrace cluster.
Furthermore, it can be used to monitor cloud centric technologies, such as AWS, Kubernetes, Openshift, or other cloud systems.

\textbf{End-to-End (E2E) testing}\cite{end-to-end-integration-testing-design} is the process of testing a software system.
As the name suggests, this technique tests a software system from one end, i.e., a hypothetical user, across user-system interaction, to the other end, i.e., the results of a feature.
For example, buying an item on an online shop can be e2e tests.
First, a test program may open a browser and navigate to the website.
It then adds certain items to a cart, evaluating the shown total on the site.
It then removes and re-adds certain items while evaluating the subtotal.
After the purchase is completed, it checks that the correct entries where made in the datastore of the web shop and that all transactions were made correctly and completely.

\textbf{Vault}\cite{vault}

\textbf{Permission bits}

\textbf{Hyper Text Transfer Protocol}

\textbf{Representational State Transfer}

\textbf{SHA Hashes}

\textbf{Virtual machines}

\section{Kubernetes}\label{sec:kubernetes}

A large part of this thesis concerns itself with a technology called \textbf{Kubernetes}.
Kubernetes is a complex software system and care must be taken to establish different parts and concepts of it.
This section covers the basic information needed to understand how it works and what certain terminology means in its context.

First, \textbf{Images} and \textbf{Containers}\cite{docker-image} act as the base of Kubernetes.
An image can be seen as a template to a container.
It includes instructions to be run to create a container.
These instructions can range from simple commands to complex, multistep build instructions.
Any image can also serve as the starting point for a new image.
This allows the creation of complex systems that can be instantiated as containers at any time.

A container\cite{what-are-linux-containers} is a process that runs on an operating system (OS), which is mostly isolated from the host system and other processes that run on it.
In a way, they act as virtual machines.
However, they do not contain an OS themselves, but make use of the host OS through abstraction layers.
The advantage of not having to virtualize the OS is usually a better performance compared to virtual machines.
On the other hand, due to them not being completely isolated from the host OS, they have a larger impact on a system's security.

A Kubernetes system is structured as a \textbf{cluster} and therefore called a ``Kubernetes cluster''.
It is called a cluster because it is made of a cluster of \textbf{node}s.
In the context of Kubernetes, a node is computer that serves as a part of a cluster.

The smallest clusters typically have three nodes.
One control node and multiple worker nodes.
Although it is possible to have clusters with only one or two nodes, but they are mostly only used for development and debugging purposes.
Bigger clusters and clusters that are made for software systems with high availability requirements can also have multiple control nodes for redundancy.
In such cluster, worker nodes interact with one control node at a time, with other control nodes taking over if one fails.

\textbf{Pods}

\textbf{ReplicaSets}

\textbf{Deployments}

\textbf{StatefulSets}

\textbf{Side-car- or init-containers}

\textbf{Webhooks}

\textbf{Manifests}

\textbf{Helm charts}

\textbf{API}

\textbf{Operators}

\textbf{Operator Lifecycle Management (OLM)}

\textbf{AWS, GKE, and OpenShift}