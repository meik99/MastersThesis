\chapter{Technologies and concepts}\label{ch:technologies-used}

An overview of the technologies that have been used is given in this section.
The name and, if applicable, the acronym of a technology is stated followed by a short description of it.

\textbf{Concourse} is a CI/CD system that can be used to automate certain tasks and group them into a pipeline.
Tasks can then be run either concurrently or sequentially in the context of the pipeline they are grouped in.
In addition, resources to be used can be defined and added to tasks, so they are available to them.
Such resources can be files, online resources, storage locations or any number of information a task may need.
Concourse is used to implement the pipeline answering \textit{Q2}.

\textbf{YAML and JSON} are data formats.
They define the structure of information that makes it easier to transform and read the information in an automated manner.

\textbf{Golang (Go)} is a programming language.
It is the language of choice that is used to implement a tool to answer \textit{Q1}.

\textbf{BitBucket} is a system that utilizes \textbf{Git} to store project files.
Git is used to version a project and make it easier for a team of people to simultaneously work on a project.
A project that is worked on in this manner is commonly called a repository.
BitBucket itself then stores the repository online to ease access to it.

\textbf{OneAgents and ActiveGates} are Dynatrace proprietary software systems.
The Dynatrace OneAgent\cite{oneagents} is a software system to monitor other systems and collect metrics of them.
These metrics can range from memory or cpu usage to more complex ones such as network connections or amount of specific function calls.
Two different kinds of OneAgents exist, a host agent and a special agents.
The host agents purpose is to monitor the host of any given system, for example, a server on which a web application is deployed on.
Then, for different technologies, different special agents exist.
A special agent's purpose is to monitor an application, such as the web application mentioned earlier.

An ActiveGates\cite{activegates} main purpose is to ac as a proxy between OneAgents and the Dynatrace cluster.
It can be used to buffer information sent by a OneAgent inside a local network, before uploading it to the main Dynatrace cluster.
Furthermore, it can be used to monitor cloud centric technologies, such as AWS, Kubernetes, Openshift, or other cloud systems.

\textbf{End-to-End (E2E) testing}\cite{end-to-end-integration-testing-design} is the process of testing a software system.
As the name suggests, this technique tests a software system from one end, i.e., a hypothetical user, across user-system interaction, to the other end, i.e., the results of a feature.
For example, buying an item on an online shop can be e2e tests.
First, a test program may open a browser and navigate to the website.
It then adds certain items to a cart, evaluating the shown total on the site.
It then removes and re-adds certain items while evaluating the subtotal.
After the purchase is completed, it checks that the correct entries where made in the datastore of the web shop and that all transactions were made correctly and completely.

\textbf{Vault}\cite{vault} is a system created by HashiCorp to manage credentials.
It offers a user interface as well as a backend to store secrets, en- and decrypt them as wells as user management.
The backend is accessible either through the provided client or the provided API.
This API is used by Concourse to retrieve secrets during pipeline runs.

The \textbf{binary number system} represents numbers in base-2.
This means, every number is represented as a series of ones and zeros, with every place representing a power of two.
E.g., $110$, in binary, equates to $1 * 2^2 + 1 * 2^1 + 0 * 2^0 = 6$ in the decimal system.
A single place in the binary system is called a bit, which can have a value of zero or one.

\textbf{Permission bits}\cite{unix-file-permissions} are used by unix file systems to control access to files.
Each file has three groups with three bits each.
Every bit signifies if a certain action, reading, writing, or executing, can be taken for a file.
The groups determine what the file owner, a group member or any other user, can do.
For example:

A file's permission bits are set to $7$, $4$, and $2$, or $742$ as a shorthand notation.
Note here that $742$ does not represent the number sevenhundredfortytwo, but the decimal value of each group as a shortened form.
That means, the file owner can do everything with the file, because the number $7$ in binary is represented as $111$.
Therefore, all permission bits are set.
Next, the number $4$ represents what the group of users that owns the file can do.
A $4$ in binary is represented as $100$.
That means, members of the group may read the file, but may neither write nor execute it.
Finally, the $2$ represents what every user in the system can do.
Since $2$ is represented as $010$, the permission bit for writing a file is set.
So every user, that is neither the file owner nor a member of the aforementioned group, may write a file, but neither read nor execute it.

The \textbf{Hyper Text Transfer Protocol (HTTP)}\cite{http-rfc} is a protocol to transmit hypertext or non-linear text.
It is most commonly used to transmit web pages from a server to a web client, but can be used for other applications as well.
For the purposes of this thesis, HTTP defines how a request to a web server looks like.
Different types of requests exist, the most common ones are \textit{GET}, \textit{POST}, \textit{PUT}, and \textit{DELETE}.
As per the protocols specification, a \textit{GET} request is used to retrieve a resource from the server.
\textit{POST} requests creates or processes resources, while \textit{PUT} is used to replace or update a resource.
Finally, \textit{DELETE} requests are used to remove a resource from the server.
Requests may also include a \textit{body}, i.e., additional information that can be interpreted by a backend.
Note that the protocol is larger than stated here, but summarizing the whole of the protocol is out of scope of this thesis.

\textbf{Representational State Transfer}\cite{extending-representation-state-transfer} is a way to build decentralized systems which depend on sharing resources.
A REST based system commonly builds on top of HTTP to request, process and transmit resources.

\textbf{SHA Hashes}\cite{cryptographic-hash-functions,sha-hashes} are a family of algorithms to create a fingerprint for a file.
The main advantages of hashes is that they are irreversible.
For every file, a hash can be created which identifies a file, but from the hash it is not possible to recreate the file.
Furthermore, the latest iterations of hash algorithms, which are the \textit{SHA} family.
They have a very low possibility of the fingerprints, or digests, of two different files accidentally matching, also known as hit collision.
Therefore, if, for example, a website offers a file for download as well as a hash to verify the file's integrity, the file's integrity is not compromised if they match.

\textbf{Virtual machines (VM)}\cite{what-is-a-virtual-machine} is a virtual operating system, which is emulated on top of a real operating system.
The virtual operating system can vastly differ from the host system.
For example, a Windows system can be emulated as a virtual machine inside a Linux host system.
Going further, since a VM can be completely decoupled from the host system, a host system does not necessarily need an operating system of its own.
Using an abstraction layer called a \textit{Hypervisor}, VMs can be started on a computer without the need for a full operating system, saving system resources.
VMs, however, run slower than physical machines.

\section{Kubernetes}\label{sec:kubernetes}

A large part of this thesis concerns itself with a technology called \textbf{Kubernetes}.
Kubernetes is a complex software system and care must be taken to establish different parts and concepts of it.
This section covers the basic information needed to understand how it works and what certain terminology means in its context.

First, \textbf{Images} and \textbf{Containers}\cite{docker-image} act as the base of Kubernetes.
An image can be seen as a template to a container.
It includes instructions to be run to create a container.
These instructions can range from simple commands to complex, multistep build instructions.
Any image can also serve as the starting point for a new image.
This allows the creation of complex systems that can be instantiated as containers at any time.

A container\cite{what-are-linux-containers} is a process that runs on an operating system (OS), which is mostly isolated from the host system and other processes that run on it.
In a way, they act as virtual machines.
However, they do not contain an OS themselves, but make use of the host OS through abstraction layers.
The advantage of not having to virtualize the OS is usually a better performance compared to virtual machines.
On the other hand, due to them not being completely isolated from the host OS, they have a larger impact on a system's security.

A Kubernetes system is structured as a \textbf{cluster} and therefore called a ``Kubernetes cluster''.
It is called a cluster because it is made of a cluster of \textbf{node}s.
In the context of Kubernetes, a node is computer that serves as a part of a cluster.

The smallest clusters typically have three nodes.
One control node and multiple worker nodes.
Although it is possible to have clusters with only one or two nodes, but they are mostly only used for development and debugging purposes.
Bigger clusters and clusters that are made for software systems with high availability requirements can also have multiple control nodes for redundancy.
In such cluster, worker nodes interact with one control node at a time, with other control nodes taking over if one fails.

\textbf{Pods}

\textbf{ReplicaSets}

\textbf{Deployments}

\textbf{StatefulSets}

\textbf{Side-car- or init-containers}

\textbf{Webhooks}

\textbf{Manifests}

\textbf{Helm charts}

\textbf{API}

\textbf{Operators}

\textbf{Operator Lifecycle Management (OLM)}

\textbf{AWS, GKE, and OpenShift}