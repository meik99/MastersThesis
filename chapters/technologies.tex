\chapter{Technologies and concepts}\label{ch:technologies-used}

An overview of the technologies that have been used is given in this section.
The name and, if applicable, the acronym of a technology is stated followed by a short description of it.
Then follows the reason for its involvement in this project.

\textbf{Concourse} is a CI/CD system that can be used to automate certain tasks and group them into a pipeline.
Tasks can then be run either concurrently or sequentially in the context of the pipeline they are grouped in.
In addition, resources to be used can be defined and added to tasks, so they are available to them.
Such resources can be files, online resources, storage locations or any number of information a task may need.
Concourse is used to implement the pipeline answering \textit{Q2}.

\textbf{YAML and JSON} are data formats.
They define the structure of information that makes it easier to transform and read the information in an automated manner.

\textbf{Golang (Go)} is a programming language.
It is the language of choice that is used to implement a tool to answer \textit{Q1}.

\textbf{BitBucket} is a system that utilizes \textbf{Git} to store project files.
Git is used to version a project and make it easier for a team of people to simultaneously work on a project.
A project that is worked on in this manner is commonly called a repository.
BitBucket itself then stores the repository online to ease access to it.

\textbf{OneAgents and ActiveGates}

\textbf{End-to-End (E2E) testing}

\textbf{Vault}

\section{Kubernetes}\label{sec:kubernetes}

\textbf{Operators}

\textbf{AWS, GKE, and OpenShift}

\textbf{API}

\textbf{Images}

\textbf{Containers}

\textbf{Pods}

\textbf{ReplicaSets}

\textbf{Deployments}

\textbf{StatefulSets}

\textbf{Webhooks}

\textbf{Side-car- or init-containers}

\textbf{Manifests}

\textbf{Helm charts}

\textbf{Operator Lifecycle Management (OLM)}