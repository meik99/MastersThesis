\chapter{Technologies and concepts}\label{ch:technologies-used}

An overview of the technologies that have been used is given in this section.
The name and, if applicable, the acronym of a technology is stated followed by a short description of it.

\textbf{Golang (Go)}~\cite{golang} is a programming language.
It is the language of choice that is used to implement the DTF.

\textbf{BitBucket}~\cite{bitbucket} is a system that utilizes \textbf{Git}~\cite{git} to store project files.
Git is used to version a project and make it easier for a team of people to simultaneously work on a project.
A project that is worked on in this manner is commonly called a repository.
BitBucket itself then stores the repository online to ease access to it.

\textbf{Vault}~\cite{vault} is a system created by HashiCorp to manage credentials.
It offers a user interface as well as a backend to store secrets, en- and decrypt them as wells as user management.
The backend is accessible either through the provided client or the provided API.
This API is used by Concourse to retrieve secrets during pipeline runs.

\textbf{Permission bits}~\cite{unix-file-permissions} are used by unix file systems to control access to files.
Each file has three groups with three bits each.
Every bit signifies if a certain action, reading, writing, or executing, can be taken for a file.
The groups determine what the file owner, a group member or any other user, can do.
For example:

A file's permission bits are set to $7$, $4$, and $2$, or $742$ as a shorthand notation.
Note here that $742$ does not represent the number sevenhundredfortytwo, but the decimal value of each group as a shortened form.
That means, the file owner can do everything with the file, because the number $7$ in binary is represented as $111$.
Therefore, all permission bits are set.
Next, the number $4$ represents what the group of users that owns the file can do.
A $4$ in binary is represented as $100$.
That means, members of the group may read the file, but may neither write nor execute it.
Finally, the $2$ represents what every user in the system can do.
Since $2$ is represented as $010$, the permission bit for writing a file is set.
So every user, that is neither the file owner nor a member of the aforementioned group, may write a file, but neither read nor execute it.

\textbf{Representational State Transfer}~\cite{extending-representation-state-transfer} is a way to build decentralized systems which depend on sharing resources.
A REST based system commonly builds on top of HTTP to request, process and transmit resources.
