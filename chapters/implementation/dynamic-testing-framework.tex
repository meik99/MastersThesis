\section{Dynamic Testing Framework}\label{sec:dynamic-testing-framework}

The Dynamic Testing Framework (DTF) is a tool created to generate a test suite.
Its input are test definitions and a defined similarity.
Its output is a pipeline, that runs a subset of tests as defined by test definitions and similarities.
As discussed in Chapter~\ref{ch:background}, it uses a similarity based algorithm, to choose a subset of tests to run.

The purpose of this tool is to generate a pipeline based on a set of tests.
Tests are defined and their similarity to each other are set.
A subset of the tests is then selected randomly, based on the similarity, and a pipeline with a task structure is generated from them.
As a main goal, the spread of tested features should be as wide as possible, since testing every feature with every setting in every combination is not feasible.

\subsection{Test definitions}\label{subsec:test-definitions}

A test definitions defines what should be run inside a test.
For each test definition, two sets of files will be generated, one set for Kubernetes and one for Openshift.
A set consists of two files, a script that is run by Concourse and a Concourse specific task configuration.

\pagebreak

The generated files and folder structures are specific to Concourse at the time of writing and other systems are currently not supported.
In the same manner, the template files are embedded and cannot be changed without recompiling the tool.
These flaws are planned to be fixed in the future\footnote{They were indeed fixed, however only at a later date and outside of the scope of this thesis}.
A test definition may look as follows.

\begin{verbatim}
kind: TestDefinition
name: A simple test
setup:
  - /path/to/setup.sh
  - /path/to/setup-2.sh
test:
  - /path/to/step-1.sh
  - /path/to/step-2.sh
teardown:
  - /path/to/teardown.sh
  - /path/to/teardown-2.sh
environment:
  ENV_VAR: value
resource:
  resource-name: resource-reference
  another-resource: another-resource
\end{verbatim}

First, the \textbf{kind} property must exist and must be set exactly to \verb|TestDefinition|.
This kind is used to find test definitions in folders and combined YAML files.
A source of a test definition may be a folder or a file.
The file can either consist of one or multiple test definitions or a combination of objects.
In order to find every test definition, if a folder is passed as the source, every file will be parsed as a YAML file.
If it can be parsed successfully, every object is iterated over.
If the object contains this ``kind'' property and its value is \verb|TestDefinition|, it is recognized as such.

\pagebreak

The \textbf{name} property defines how the test definition is referenced in the similarity file and how the name in the test configuration for Concourse will look like.
This property must be case-insensitive unique, i.e., the name \verb|A test| is the same as \verb|a test|.
Therefore, there must not be two tests which, in their name, differ only in case.
This is because the name in the Concourse configuration will be in a normalized form.
In this form, spaces are replaced with dashes and all text will be in lowercase.
Going back to the previous example, both names will result in the string \verb|a-test| when normalized.

The \textbf{setup} property allows for specifying scripts which are run before the test steps are run.
Practically speaking, it would have the same effect as moving the scripts to the top of the \textbf{test} list.
However, this way allows for a better folder structure in a test repository, since the files for test steps can be more cleanly seperated from utility scripts.
Similarly, the \textbf{teardown} property allows for specifying scripts which should be run after the test steps were run.

In the \textbf{test} property, the paths to test scripts can be defined.
The scripts are executed in the same order as they are supplied.

The \textbf{environment} and \textbf{resource} properties can be used to specify environment variables or Concourse resources that should be passed to the test task.
The value will be passed to the Concourse configuration exactly as stated in the test definition, i.e., if \verb|((something))| is passed, \verb|((something))| will be passed to Concourse, which in turn will make a Vault lookup.
With \verb|{{something}}| a parameter from a params-file can be referenced.
Simply passing \verb|something| will pass the value \verb|something|.

For the \textbf{resource} property, the key of the map is used to reference resources of the Concourse pipeline.
The value is not used for now, but is planned to be used to rename the resource and pass it to tasks under different names.
Finally, note that only the \textbf{kind}, \textbf{name}, and \textbf{test} properties are mandatory.
All other properties are optional.

DTF will look for test definitions in its working directory.
This behaviour can be changed by supplying the appropriate flags.
For every supplied source, it is checked whether it is a directory or a file.

If a directory was supplied, the files in the directory are parsed.
Directories inside supplied directories will not be read.
In other words, directories are not traversed recursively and child directories have to be supplied explicitly.

Each file found, be it a file directly supplied or found in a directory, is parsed as a YAML file.
\begin{itemize}
  \item If it cannot be parsed as a YAML file, a warning is shown.
  \item If it can be parsed as a YAML, all objects in the file are iterated upon.
  \item If an object has the property \textbf{kind}, and it is set to \textbf{TestDefinition}, it will be handled as a test definition.
  \item If any mandatory property, i.e., \textbf{kind}, \textbf{name}, or \textbf{test}, are missing, an error is printed.
\end{itemize}

\subsection{Similarities}\label{subsec:similarities}

A similarity defines how similar a test is to other tests.
This is the implementation of the concept introduced in Section~\ref{sec:similarities}.
It is used to select tests for a pipeline with a greater amount of variety than pure randomness.
First, a random test is picked and added to the list of tests.
Then, the list of all other possible tests is sorted by descending similarity, such that the most dissimilar test is at the beginning of the list.
This test is then removed and added to the picked tests.
Then, the list of tests is again sorted, but in reference to the last picked tests.
By picking the test most dissimilar to the previous one, the variety can be increased compared to choosing tests at random.

A similarity file may look as follows.
Note well that the similarity objects do not have a \verb|kind| property.
Therefore, it is not possible to mix similarities with other types of YAML objects.
This is done on purpose, to force a single file having all similarities neatly organized inside a project.

\pagebreak

\begin{verbatim}
name: Another simple test
similarity:
  - name: A similar test
    degree: similar
  - name: A dissimilar test
    degree: dissimilar
exclusivity:
  - An exclusive test
dependency:
  - A simple test
---
name: A similar test
similarity:
  - name: Another simple test
    degree: similar
  - name: A dissimilar test
    degree: neutral
---
name: A dissimilar test
similarity:
  - name: A non existent test
    degree: equal
  - name: A different non existent test
    degree: opposite
\end{verbatim}

The \textbf{name} property is used to reference a test.
A value passed with this property must match the name of a test definition.
If no test definition with the same name exists, the program will print an error.

The \textbf{similarity} property lists the similarities to other test cases.
Every list entry has a \textbf{name} property, referencing another tests, and a \textbf{degree} of similarity.
The possible entries for the degree are as follows, starting with the highest dissimilarity.
The degrees must be given as stated below and are case-sensitive, due to the way the used YAML parser works.

\pagebreak

\begin{itemize}
  \item opposite
  \item dissimilar
  \item neutral
  \item similar
  \item same
\end{itemize}

If a different test is not referenced in the \textbf{similarity} list, the degree defaults to neutral or to the degree that the other test references.
For example:
\begin{itemize}
  \item The test \verb|A| does not reference test \verb|B| and \verb|B| does not reference test \verb|A|. \verb|A| and \verb|B| are neutral to each other.
  \item The test \verb|A| references test \verb|B| as \verb|dissimilar|, but \verb|B| does not reference test \verb|A|. \verb|A| and \verb|B| are dissimilar to each other.
  \item The test \verb|A| does not reference test \verb|B|, but \verb|B| reference test \verb|A| as \verb|dissimilar|. \verb|A| and \verb|B| are dissimilar to each other.
  \item The test \verb|A| references test \verb|B| as \verb|dissimilar|, but \verb|B| references test \verb|A| explicitly as \verb|neutral|. An error is thrown, because the similarities are inconsistent.
\end{itemize}

The properties \textbf{exclusivity} and \textbf{dependency} are planned to be used with tests that either can or can not run on the same cluster.
However, this behaviour is not yet supported.
It will either be implemented in the future or removed if deemed unnecessary.
