\subsection{Prepare OpenShift infrastructure}\label{subsec:prepare-openshift-infrastructure}

The purpose of this task it to prepare an S3 bucket for the Openshift installer to place a state file in.
This task creates an S3 bucket to place openshift installer state into.
It uses a python script to do so.
This script takes the parameters shown in table \ref{tab:parameters-to-create-an-s3-bucket}.

\begin{table}[H]
    \centering
    \caption{Parameters to create an S3 bucket}
    \label{tab:parameters-to-create-an-s3-bucket}
    \begin{tabular}{p{0.3\linewidth}|p{0.6\linewidth}}
        Parameter & Expected value \\
        \hline
        \verb| --cluster-name | & The name of the cluster which will be created.
            It is usually the same name as the pipeline. \\
        \verb| --bucket-name | & The name the created bucket should have. \\
        \verb| --bucket-region | & The AWS region the bucket should be created in. \\
        \verb| --remove | & A flag to remove an existing bucket again \\
    \end{tabular}
\end{table}

The \verb| boto3 | python library is then used to connect to AWS S3 service.
If the given bucket name does not exist, it is created and an empty file is uploaded as a state file.
If the given bucket name does exist, but a state file is missing, an empty file is created and uploaded.
