\subsection{prepare-openshift-infrastructure}\label{subsec:prepare-openshift-infrastructure}

This task creates an S3 bucket to place openshift installer state into.
It uses a python script, located at [/scripts/openshift/s3_setup.py](../../scripts/openshift/s3_setup.py), to do so.
This script takes the following parameters.

| Parameter         | Expected value                                                                                                      | Default |
| ----------------- | ------------------------------------------------------------------------------------------------------------------- | ------- |
| `--cluster-name`  | The name of the cluster which will be created. It is usually the same name as the pipeline.                         |         |
| `--bucket-name`   | The name the created bucket should have.                                                                            |         |
| `--bucket-region` | The AWS region the bucket should be created in. Like everything and your doorbell, it is usually set to `us-east-1` |         |
| `--remove`        | A flag to remove an existing bucket again                                                                           | N.A.    |

The `boto3` python library is then used to connect to AWS S3 service.
If the given bucket name does not exist, it is created and an empty file is uploaded as a state file.
If the given bucket name does exist, but a state file is missing, an empty file is created and uploaded.

\subsubsection{Purpose}\label{subsubsec:prepare-openshift-infrastructure-Purpose}

The purpose of this task it to prepare an S3 bucket for the Openshift installer to place a state file in.
