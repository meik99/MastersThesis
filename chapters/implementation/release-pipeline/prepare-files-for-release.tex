\textbf{Prepare files for release}

The task generates the CSV bundles for Openshift and Kubernetes OLM systems.
Since most files and information is available, minor changes like setting version properties are also done.

This task is used to generate the CSV bundles for the OLM releases as well as setting minor details in other files.
To do so, it first needs to install the Operator SDK.
The version to download can be configured with the \verb|operator_sdk_version| parameter.
If the parameter is not set, the latest version available is downloaded.

\pagebreak

Then the release branch is checked out of the target repository.
From this branch, the CSV branch is created.
In order to get a clean branch, it is first deleted if it exists and the deletion is force pushed afterwards.
Force deleting the branch will close open pull requests that are based on this branch.

Afterwards, the following changes are made. \\
In \verb|<repository>/config/helm/chart/default/Chart.yaml|:
\begin{itemize}
    \item The \verb| version | property is set to the version which is going to be released
    \item The \verb| appVersion | property is set to the version which is going to be released
    \item Setting this property is done using a library, which does not preserve comments.
    Therefore, the license text is re-added to the top afterwards
\end{itemize}

In \verb|<repository>/config/helm/schema.yaml|
\begin{itemize}
    \item The \verb| x-google-marketplace.publishedVersion | property is set to the version which is going to be released
    \item The \verb| x-google-marketplace.publishedVersionMetadata.releaseNote | property is set to the generated changelog
    \item Again, the license text is re-added afterwards
\end{itemize}

In \verb|dynatrace-operator/config/helm/chart/default/templates/application.yaml|
\begin{itemize}
    \item The \verb|version| property is set to the version which is going to be released
    \item This file is a Helm template, which has some extended notations compared to a normal YAML file.
        Therefore, the same library used before cannot be used here to set this property.
    \item In order to mitigate this problem, the replacement is done using \verb|sed| and the license text does not need to be re-added.
\end{itemize}

Then, the CSV bundles are generated.
Once for Kubernetes, another time for Openshift.
A target of the Makefile included in the dynatrace-operator repository is used to generate these files.
The following parameters are set to make sure the image is correctly replaced for each platform.

For Kubernetes, the following parameters are set.

\begin{verbatim}
IMG="docker.io/dynatrace/dynatrace-operator:v${version}"
MASTER_IMAGE="docker.io/dynatrace/dynatrace-operator:v${version}"
BRANCH_IMAGE="docker.io/dynatrace/dynatrace-operator:v${version}"
OLM_IMAGE="docker.io/dynatrace/dynatrace-operator:v${version}"
\end{verbatim}

For Openshift, the following parameters are set.

\begin{verbatim}
IMG="registry.connect.redhat.com/dynatrace/dynatrace-operator:v${version}"
BRANCH_IMAGE="registry.connect.redhat.com/dynatrace/dynatrace-operator:v${version}"
MASTER_IMAGE="registry.connect.redhat.com/dynatrace/dynatrace-operator:v${version}"
OLM_IMAGE="registry.connect.redhat.com/dynatrace/dynatrace-operator:v${version}"
\end{verbatim}

Finally, the generated CSV files are finalized by a python script.
This script main purpose is to set the correct date for the \verb|metadata/annotations/createdAt| property in the main CSV files.
It also sets the \verb|spec/replaces| field to the version of the last GitHub release, however, this value is only used as a fallback.
This property is later overwritten to depend on the release CSV files instead of the released GitHub versions.

After all changes were made, they are pushed to the CSV branch.
Finally, a pull request from the CSV branch to the release branch is created.

