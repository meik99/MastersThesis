\subsection{Release RedHat marketplace}\label{subsec:release-redhat-marketplace}

This task releases the CSV bundle for Openshift to the RedHat marketplace repository.
Releasing the CSV files to this repository makes the DTO available on the RedHat Marketplace for Operators.
The purpose of this task is to execute all steps necessary to release the CSV bundle to the RedHat Marketplace for Operators.
This includes forking the repository and following the RedHat Marketplace guidelines.
A pull request or draft pull request is not done to avoid operating on a foreign repository with automatic tasks.

It first clones the DTO repository and then creates a fork of the RedHat marketplace repository.
Forking is done by a python script which takes the arguments shown in table\ \ref{tab:parameters-to-fork-the-redhat-marketplace-repository}.

\begin{table}[H]
    \centering
    \caption{Parameters to fork the RedHat marketplace repository}
    \label{tab:parameters-to-fork-the-redhat-marketplace-repository}
    \begin{tabular}{p{0.3\linewidth}|p{0.6\linewidth}}
        Parameter & Expected value \\
        \hline
        \verb|--token| & A personal access token for GitHub.
            It should have the following scopes \verb|admin:org, delete_repo, repo, workflow|.
            The same token can be used by multiple tasks if these scopes are set. \\
        \verb|--original-repository| & The original repository which should be fork. \\
        \verb|--forked-repository| & The name the forked repository should have. \\
        \verb|--output-file| & The path to which the metadata of the fork is written to. \\
        \verb|--retries| & How often the scripts should retry requests should they fail. \\
        \verb|--retry-timeout| & How long, in seconds, the script waits between request retries. \\
    \end{tabular}
\end{table}

After parsing the arguments, an instance of \verb|GitHubApi| is created using \verb|{forked-repository}| and \verb|{token}|.
This class implements some REST requests GitHub's API provides.
First, the \verb|exists| function is called, which returns \verb|True| if the forked repository already exists.

\pagebreak

To ensure a consistent environment, forked repositories are deleted if they already exist.
Therefore, if the \verb|exist| function returns true, the fork is deleted by calling \verb|delete_repository|, which returns \verb|True| if the request succeeds or \verb|False| if it doesn't.
Additionally, the response from the GitHub API is also returned.

If the forked repository does not exist or has successfully been deleted, it is recreated.
This is done by calling the \verb|fork| function and passing \verb|{original-repository}| as a parameters.
On a success, a check is made, if the forked repository has the expected name \verb|{forked-repository}|.
If it does not, the repository is renamed by instantiating \verb|GitHubApi| with the newly forked repository and calling the \verb|rename| function.
The metadata of the forked repository is then written to \verb|{output-file}|.

In this forked repository, first, the latest released version is determined.
A python script iterates over the directories in the \verb|dynatrace-operator-rhmp| folder.
The directories are named after the released version, i.e., \verb|0.2.2|, \verb|0.3.0|, etc.
The latest version is then saved in a variable \verb|last_version|.
If no version exist, it defaults to an empty string.

Afterwards, a new folder for the version that is to be released is created.
The folders for the manifest and metadata are then populated with the generated CSV bundles respectively.
In the CSV file of the version to be released, the \verb|replaces| property is then updated, if \verb|last_version| is not empty.
If the variable is empty, the property is removed.
Setting this property allows Openshift to correctly update the DTO.

Then, annotations mentioning the string ``dynatrace-operator'' are changed to read ``dynatrace-operator-rhmp'' and the CSV file is renamed from  mentioning ``dynatrace-operator'' to mentioning ``dynatrace-operator-rhmp'' as well.
This is necessary to reflect the different directory name RedHat has given its Marketplace releases.
Finally, the changes are added, committed and pushed to the forked repository.
