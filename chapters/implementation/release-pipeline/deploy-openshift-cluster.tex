\subsection{Deploy OpenShift cluster}\label{subsec:deploy-openshift-cluster}

This task deploys an Openshift cluster on AWS infrastructure.
First, it checks if a state file exists in the S3 bucket created in the previous task.
The file is then unzipped.
If an install-configuration already exists in this file, the installation was already done and the current one is skipped.

Otherwise, Openshift is installed.
First, a new install-configuration file is created, based one the \verb|{install_config}| parameter.
It specifies the AWS region and the cluster name, which are configured with the \verb|{aws_region}| and \verb|{cluster_name}| respectively, are set.

Downloading the Openshift installer is done in one of two ways.
First, by setting a installer URL.
If this URL is set, the Openshift installer is downloaded from the URL and the script assumes it is a binary.
Since changing the installer URL is not necessary for this particular pipeline, there is no parameter to set it and it is instead hard-coded.

The second way of downloading Openshift, which is supported by the pipeline, is by defining the target version, URL to the Openshift installer directory and whether unstable versions should also be considered or not.
Configuring these values is done with the \verb|{ocp_version}|, \verb|{ocp_installer_directory_url}|, and \verb|{ocp_installer_unstable}| parameters respectively.
The task then downloads the latest patch version of the installer and extracts the downloaded archive containing the installer.
After the installer has been downloaded, it is executed to create the cluster.

When the installer has finished, the deployments VPCs are queried, along with their security groups and NAT gateways.
These VPCs and gateways are then whitelisted AWS's infrastructure.
The installer state, which was written by the installer, is then zipped and uploaded to the S3 bucket.
Finally, the resulting \verb|kubeconfig| is saved as a Vault secret under the name of \verb|openshift|.

