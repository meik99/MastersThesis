# generalize-release-gke-kubernetes

This job consists of two tasks that run in parallel.
One task creates a kube-config and writes it to the Vault secret.
The second prepares the cluster itself for usage.

## fetch-kubeconfig

This task creates a kube-config for the previously deployed GKE cluster and writes it to the `kubeconfig` Vault secret.
First, it queries all necessary parameters using terraform.

* `k8s_cluster_client_cert`, representing the certificate which is be used to connect.
* `k8s_cluster_client_key`, the private key for the certificate above.
*  `k8s_cluster_ca_cert`, the certificate of the authority that issued the client certificate.
*  `k8s_endpoint`, the URL under which the cluster can be reached.

The kubeconfig is then generated using those parameters and written to the Vault secret.

### Purpose

The purpose of this task is to write a kube-config to the Vault secret to allow other tasks to connect to the cluster.

### Configuration options

The parameters can be changed in the [params-dev](../../params-dev.yaml) or [params-prod](../../params-prod.yaml) files.
The following parameters can be used to configure this task.

| Parameter          | Effect                                                               | Production default | Development default |
| ------------------ | -------------------------------------------------------------------- | ------------------ | ------------------- |
| `environment_name` | Defines what name the cluster has. Usually set to the pipeline name. | `operator-release` | `operator-release`  |

### Common error cases

During development, no common error cases were found.

## prepare-kubernetes

This task configures the GKE cluster previously deployed.
If there is no `ClusterRoleBinding` for the `cluster-admin` user, it is applied to the cluster.
Furthermore, if the `{istio_enabled}` parameter is set to `True`, it deploys Istio.

### Purpose

The purpose of this task is to configure the GKE cluster for further usage.

### Configuration options

The parameters can be changed in the [params-dev](../../params-dev.yaml) or [params-prod](../../params-prod.yaml) files.
The following parameters can be used to configure this task.

| Parameter       | Effect                                      | Production default | Development default |
| --------------- | ------------------------------------------- | ------------------ | ------------------- |
| `istio_enabled` | Defines if Istio should be deployed on the cluster. | `False`            | `False`             |

### Common error cases

During development, no common error cases were found.
