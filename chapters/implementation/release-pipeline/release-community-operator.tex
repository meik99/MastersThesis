\textbf{Release community operators}

This task releases the CSV bundle for Kubernetes to the community operators repository.
Releasing the CSV files to this repository makes the DTO available on OperatorHub.
The purpose of this task is to execute all steps necessary to release the CSV bundle to OperatorHub.
This includes forking the repository and following the OperatorHub guidelines.
A pull request or draft pull request is not done to avoid operating on a foreign repository with automatic tasks.

It first clones the DTO repository and then creates a fork of the community operators repository using a Python script
This script takes the arguments shown in table~\ref{tab:parameters-to-clone-and-fork-a-repository}.

\begin{table}[h]
    \centering
    \caption{Parameters to clone and fork a repository}
    \label{tab:parameters-to-clone-and-fork-a-repository}
    \begin{tabular}{p{0.3\linewidth}|p{0.6\linewidth}}
        Parameter & Expected value \\
        \hline
        \verb|--token| & A personal access token for GitHub.
            It should have the following scopes \verb|admin:org, delete_repo, repo, workflow|.
            The same token can be used by multiple tasks if these scopes are set. \\
        \verb|--original-repository| & The original repository which should be forked. \\
        \verb|--forked-repository| & The name the forked repository should have. \\
        \verb|--output-file| & The path to which the metadata of the fork is written to. \\
        \verb|--retries| & How often the scripts should retry requests should they fail. \\
        \verb|--retry-timeout| & How long, in seconds, the script waits between request retries. \\
    \end{tabular}
\end{table}

\pagebreak

After parsing the arguments, an instance of \verb|GitHubApi| is created using \verb|{forked-repository}| and \verb|{token}|.
This class implements some REST requests GitHub's API provides.
First, the \verb|exists| function is called, which returns \verb|True| if the forked repository already exists.
To ensure a consistent environment, forked repositories are deleted if they already exist.
Therefore, if the \verb|exist| function returns true, the fork is deleted by calling \verb|delete_repository|, which returns \verb|True| if the request succeeds or \verb|False| if it doesnt.
Additionally, the response from the GitHub API is also returned.

If the forked repository does not exist or has successfully been deleted, it is recreated.
This is done by calling the \verb|fork| function and passing \verb|{original-repository}| as a parameters.
On a success, a check is made, if the forked repository has the expected name \verb|{forked-repository}|.
If it does not, the repository is renamed by instantiating \verb|GitHubApi| with the newly forked repository and calling the \verb|rename| function.
The metadata of the forked repository is then written to \verb|{output-file}|.
In this forked repository, the following folders are created for the new version.

\begin{verbatim}
${fork_name}/operators/dynatrace-operator/${version}/manifests
${fork_name}/operators/dynatrace-operator/${version}/metadata
\end{verbatim}

\pagebreak

The Dockerfile located at \verb|{repository}/config/olm/kubernetes/bundle-${version}.|\\\verb|Dockerfile| is placed in the version's folder.
The previously mentioned folders, are then populated with the CSV bundles from \verb|{repository}/config/olm/kubernetes/${version}|\\\verb|/manifests| and \verb|{repository}/config/olm/kubernetes/${version}/metadata| respectively.
Finally, the changes are added, committed and pushed to the forked repository.
