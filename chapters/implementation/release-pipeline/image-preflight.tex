\subsection{Image preflight}\label{subsec:image-preflight}

The purpose of this task is to automatically check if necessary images are available on all registries.
Furthermore, it finds the digests of the DockerHub and RHCC images, so the images in the CSVs can be digest-pinned.

This task is used to check whether all images necessary for this release are available.
It needs the Docker config defined as a vault secret to have access to the necessary registries.
Furthermore, it uses a Docker in Docker image to try and pull the images.
Finally, it parses the digests of the images and adds them to the release information.

First \verb|jq| is installed to read the release data, because \verb|jq| is not installed by default on the Docker in Docker image.
Then, the Docker config provided is written to \verb|~/.docker/config.json| to make it available to Docker.
The command \verb|source /docker-lib.sh| imports all functions made available by the Docker in Docker image.
One of the functions is \verb|start_docker|, which is then called to start the Docker daemon.

Then the availability of the following images is checked.
\begin{itemize}
    \item \verb|registry.connect.redhat.com/dynatrace/dynatrace-operator:v${version}|
    \item \verb|gcr.io/${GCE_MARKETPLACE}/dynatrace-operator:${version}|
    \item \verb|docker.io/dynatrace/dynatrace-operator:v${version}|
\end{itemize}

Where \verb|version| is read from the release data using jq.
\verb|${GCE_MARKETPLACE}| is defined by the \verb|gcp_marketplace| parameter.
If any of those image cannot be pulled, the task fails.

The digest is then parsed for each image.
From the \verb|docker pull| command, for the DockerHub and RHCC image, the response is taken.
Since the digest can be found in the response, it is parsed using \verb|grep "Digest:"| to find the line.
Then, using shell expansion, the ``Digest: '' prefix is removed with \verb|${digest#"Digest: "}|.
This functionality is used twice and therefore put into a function called \verb|parse_digest| which takes the response from the pull command as its parameter.
