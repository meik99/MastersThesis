\pagebreak
\subsection{Release Rancher}\label{subsec:release-rancher}

This task releases the DTO helm chart to the Rancher repository for partner charts.
It first clones the DTO repository and then creates a fork of Rancher's repository.
The purpose of this task is to execute all steps necessary to release the DTO Helm charts to Rancher.
This includes forking their repository and follow their guidelines.
A pull request or draft pull request is not done to avoid operating on a foreign repository with automatic tasks.
Forking is done with a python script, which takes the arguments seen in table \ref{tab:parameters-for-forking-the-rancher-repository}.

\begin{table}[H]
    \centering
    \caption{Parameters for forking the rancher repository}
    \label{tab:parameters-for-forking-the-rancher-repository}
    \begin{tabular}{p{0.3\linewidth}|p{0.6\linewidth}}
        Parameter & Expected value \\
        \hline
        \verb|--token| & A personal access token for GitHub.
            It should have the following scopes \verb|admin:org, delete_repo, repo, workflow|.
            The same token can be used by multiple tasks if these scopes are set. \\
        \verb|--original-repository| & The original repository which should be forked. \\
        \verb|--forked-repository| & The name the forked repository should have. \\
        \verb|--output-file| & The path to which the metadata of the fork is written to. \\
        \verb|--retries| & How often the scripts should retry requests should they fail. \\
        \verb|--retry-timeout| & How long, in seconds, the script waits between request retries. \\
    \end{tabular}
\end{table}

After parsing the arguments, an instance of \verb|GitHubApi| is created using \verb|{forked-repository}| and \verb|{token}|.
This class implements some REST requests GitHub's API provides.
First, the \verb|exists| function is called, which returns \verb|True|, if the forked repository already exists.
To ensure a consistent environment, forked repositories are deleted, if they already exist.
Therefore, if the \verb|exist| function returns true, the fork is deleted by calling \verb|delete_repository|, which returns \verb|True| if the request succeeds or \verb|False| if it doesnt.
Additionally, the response from the GitHub API is also returned.

If the forked repository does not exist or has successfully been deleted, it is recreated.
This is done by calling the \verb|fork| function and passing \verb|{original-repository}| as a parameters.
On a success, a check is made, if the forked repository has the expected name \verb|{forked-repository}|.
If it does not, the repository is renamed by instantiating \verb|GitHubApi| with the newly forked repository and calling the \verb|rename| function.
The metadata of the forked repository is then written to \verb|{output-file}|.

When the fork has been created, the task continues to clone it and check out a new branch \verb|dynatrace-operator-v${version}|.
On this branch, a new directory for the version of the new release is created, committed and pushed.
Then, the contents of \verb|{repository}/config/helm/chart/default/|, i.e., the DTO Helm charts, are copied to this new directory.
In the copy of the \verb|Chart.yaml|, the icon property is then set to \verb|file://../logo.png|, to reference a local file.
Finally, the changes are added, committed and pushed to the forked repository.
