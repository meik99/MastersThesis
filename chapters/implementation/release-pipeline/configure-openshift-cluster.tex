# configure-openshift-cluster

This task, like `prepare-openshift-infrastructure` and `deploy-openshift-cluster`, was taken from the repository [pipeline-cpn-kubernetes](https://bitbucket.lab.dynatrace.org/projects/CPN/repos/pipeline-cpn-kubernetes/browse) repository.
It labels nodes and can update the openshift version, if the `{ocp_channel}` has changed.

## Purpose

Minor setup of nodes and updating Openshift.

## Configuration options

The parameters can be changed in the [params-dev](../../params-dev.yaml) or [params-prod](../../params-prod.yaml) files.
The following parameters can be used to configure this task.

| Parameter   | Effect                                                                         | Production default | Development default |  | Production default | Development default |
| ----------- | ------------------------------------------------------------------------------ | ------------------ | ------------------- |
| ocp_channel | Defines which release channel of Openshift should be used to check for update. | `stable`           | `stable`            |

## Common error cases

This task usually only fails if the S3 bucket does not exist. 
I.e., if the previous tasks were not run in the correct order or if `cleanup-openshift-infrastructure` ran in the mean time.
The correct order of tasks is as follows.

1. prepare-openshift-infrastructure
2. deploy-openshift-cluster
3. configure-openshift-cluster
4. configure-openshift-pipeline
5. Optional:
   * test-certified-operators
   * test-redhat-marketplace-operators
6. destroy-openshift-cluster
7. cleanup-openshift-infrastructure
