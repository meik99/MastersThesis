\subsection{Release GKE}\label{subsec:release-gke}

This job consists of multiple tasks.
Their goal is to build and push the deployer image for GCM, prepare a GKE cluster, install the deployer image on it and check if a basic Operator deployment succeeds.

## build-and-publish-deployer-image

This task builds the Dockerfile at `<repository>/config/helm/` and pushes the result as the deployer image.

### Purpose

The purpose of this task is to build a deployer image and publish it to GCR.

## prepare-cluster

This task prepares a GKE cluster to be able to deploy the previously built deployer image.
First, the kube-config is read from the vault secret, which is set by [/tasks/20_fetch-kubeconfig/](../../tasks/20_fetch-kubeconfig/task.sh).
Therefore, for this task, and by extension this job, to succeed, a GKE cluster must be running.
A GKE cluster can be created with [/tasks/10_create-infrastructure/](../../tasks/10_create-infrastructure/task.sh), after which `fetch-kubeconfig` must be run.

After the kube-config is read and written to `\${HOME}/.kube/config`, the GCP application CRD is applied from `https://raw.githubusercontent.com/GoogleCloudPlatform/marketplace-k8s-app-tools/master/crd/app-crd.yaml`.
This CRD is needed so GCP applications can be deployed.
If the `dynatrace` namespace or the DynaKube CRD exist on the cluster, they are deleted to ensure a clean state.
Afterwards, the namespace is created and the DynaKube CRD is applied again.

### Purpose

The purpose of this task is to prepare a GKE cluster for the deployment of deployer image.
This includes cleaning any leftover resources created by a previous DTO deployment and applying the GKE Application CRD.

### Configuration options

The parameters can be changed in the [params-dev](../../params-dev.yaml) or [params-prod](../../params-prod.yaml) files.
The following parameters can be used to configure this task.

| Parameter             | Effect                                                    | Production default             | Development default                     |
| --------------------- | --------------------------------------------------------- | ------------------------------ | --------------------------------------- |
| `operator_repository` | Defines in which repository the Dockerfile is looked for. | `Dynatrace/dynatrace-operator` | `dt-team-kubernetes/dynatrace-operator` |

### Common error cases

During development, no common error cases were found.
However, if this task fails, check if the kube-config secret exists and the content points to a running GKE cluster.

## install-deployer-image

This task uses Google's `mpdev` tool to deploy the deployer image onto the cluster mentioned above.
Since it also uses a Docker in Docker image, it first install `jq` to read the image name from the release data.
It then installs the `mpdev` tool and prepares the kube-config file as well as the GKE service account.

The `mpdev` tool can take parameters which are passed to the custom resource of the DTO.
In order to check a simple deployment, the following properties are given to the tool in a JSON format.

```json
    {
    "name": "dynatrace-operator",
    "namespace": "dynatrace",
    "apiUrl": "<a valid API url>",
    "apiToken": "<a valid api token for the given API>",
    "paasToken": "<a valid paas token for the given API>"
}
```

The API URL, the API token, as well as the PAAS token, are defined as vault secrets.
After executing the command `./mpdev install --deployer="${image}" --parameters="${parameters}"`, `mpdev` deploys the deployer image with the parameters from above.

### Purpose

This tasks purpose is to deploy the deployer image on a prepared GKE cluster.
The deployment is then monitored by the next task.

## sanity-check-deployer

This task checks the deployment job the `mpdev` tool creates for its success.
To do so, it again sets up the kube-config.
Afterwards, the condition type of the job `job/dynatrace-operator-deployer` is monitored.
If it does not reach the state `Complete` after five minutes, the task fails.

If the job reaches this state, the `.status.succeeded` flag is checked.
If the job did not succeed, the task fails.
If the job succeeded, the pod states are checked by [/scripts/check_pod_status.py](../../scripts/check_pod_status.py).
If the pod's states are `Terminated` and the reason is `Error`, the task fails.

If the job completes successfully and all DTO pods are running afterwards, the task succeeds and the deployer image can be assumed to work.

### Purpose

This tasks purpose is to make a quick test whether the deployer image can deploy the DTO successfully.
