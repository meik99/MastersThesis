\subsection{Release GKE}\label{subsec:release-gke}

This job consists of multiple tasks.
Their goal is to build and push the deployer image for GCM, prepare a GKE cluster, install the deployer image on it and check if a basic Operator deployment succeeds.

\subsubsection{Building and pushing a deployer image}\label{subsubsec:build-and-publish-deployer-image}

The purpose of this task is to build a deployer image and publish it to GCR.
This task builds the Dockerfile at \verb|<repository>/config/helm/| and pushes the resulting image to GCR as the deployer image.

\subsubsection{Preparing a cluster}\label{subsubsec:prepare-cluster}

The purpose of this task is to prepare a GKE cluster for the deployment of deployer image.
This includes cleaning any leftover resources created by a previous DTO deployment and applying the GKE Application CRD.
This task prepares a GKE cluster to be able to deploy the previously built deployer image.
First, the kube-config is read from the vault secret, which is set by the task of section \ref{subsubsec:fetch-kubeconfig}.
Therefore, for this task, and by extension this job, to succeed, a GKE cluster must be running.
A GKE cluster can be created with the task of section \ref{subsec:create-release-gke-infrastructure}.

After the kube-config is read and written to \verb|\${HOME}/.kube/config|, the GCPs application CRD is applied from \url{https://raw.githubusercontent.com/GoogleCloudPlatform/marketplace-k8s-app-tools/master/crd/app-crd.yaml}.
This CRD is needed so GCP applications can be deployed.
If the \verb|dynatrace| namespace or the DynaKube CRD exist on the cluster, they are deleted to ensure a clean state.
Afterwards, the namespace is created and the DynaKube CRD is applied again.

\subsubsection{Installing the deployer image}\label{subsubsec:install-deployer-image}

This tasks purpose is to deploy the deployer image on a prepared GKE cluster.
The deployment is then monitored by the next task.
This task uses Google's \verb|mpdev| tool to deploy the deployer image onto the cluster mentioned above.
Since it also uses a Docker in Docker image, it first installs \verb|jq| to read the image name from the release data.
It then installs the \verb|mpdev| tool and prepares the kube-config file as well as the GKE service account.

The \verb|mpdev| tool can take parameters which are passed to the custom resource of the DTO.
In order to check a simple deployment, the following properties are given to the tool in a JSON format.

\begin{verbatim}
{
    "name": "dynatrace-operator",
    "namespace": "dynatrace",
    "apiUrl": "<a valid API url>",
    "apiToken": "<a valid api token for the given API>",
    "paasToken": "<a valid paas token for the given API>"
}
\end{verbatim}

The API URL, the API token, as well as the PAAS token, are defined as vault secrets.
After executing a command, \verb|./mpdev install|, with the parameters \verb|--deployer="${image}"| and \verb|--parameters="${parameters}|, \verb|mpdev| deploys the deployer image with the parameters from above.

\subsubsection{Sanity checking the deployer deployment}\label{subsubsec:sanity-check-deployer}

This tasks purpose is to make a quick test whether the deployer image can deploy the DTO successfully.
This task checks the deployment job the \verb|mpdev| tool creates for its success.
To do so, it again sets up the kube-config.
Afterwards, the condition type of the job \verb|job/dynatrace-operator-deployer| is monitored.

\begin{itemize}
    \item If it does not reach the state \verb|Complete| after five minutes, the task fails.
    \item If the job reaches this state, the \verb|.status.succeeded| flag is checked.
    \item If the job did not succeed, the task fails.
    \item If the job succeeded, the pod's states are checked by a Python script.
    \item If the pod's states are \verb|Terminated| and the reason is \verb|Error|, the task fails.
    \item If the job completes successfully and all DTO pods are running afterwards, the task succeeds and the deployer image can be assumed to work.
\end{itemize}
