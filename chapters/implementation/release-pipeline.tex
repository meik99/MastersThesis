\section{Release pipeline}\label{sec:release-pipeline}

In this section, the implementation of the release pipeline is explained.
For each task of the pipeline, the following questions are answered.

\begin{itemize}
    \item What does the task do and how does it do it?
    \item Why does it do it?
\end{itemize}

Futhermore, this section is split into three parts.
First, the main release tasks.
These tasks are the steps necessary to generate a release for the DTO and release it on the mentioned marketplaces.
Then, there are tasks specific for GKE and OpenShift.
These tasks are used to create either a GKE or an OpenShift infrastructure to test the release on.

Note that there is a distinction between the tests for the DTO generated by the DTF and the tests using this infrastructure.
The former tests the DTO itself.
They test if the latest changes to the DTO did not introduce any faults to established functionality.
The latter test if the release artifacts themselves, e.g., the Helm chart, work as intended.

\subsection{Main release tasks}\label{subsec:main-release-tasks}

\begin{figure}[H]
    \centering
    \includegraphics[width=\textwidth]{img/implementation/release-pipeline.drawio}
    \caption{Implemented release pipeline.}
    \label{fig:implemented-release-pipeline}
\end{figure}

This section discusses the tasks implemented as part of the main release pipeline.
In Figure~\ref{fig:implemented-release-pipeline} the final implementation of this pipeline part is depicted.
The boxes contain the names of the tasks, while the freestanding text depicts the resources they either consume or produce.
Most of them consume the DTO's repository and the release-data generated by the previous task.
They then generate a new version of the release-data, amended with the data generated by the tasks.

\subsection{Finding the latest release branch}\label{subsec:finding-the-latest-release-branch}

The purpose of this task is to automatically find the branch to operate on, since pipeline is designed to generate releases for the next version of the operator.
Since the script already parses version information, the branch names which are used by other tasks to generate CSV files and Helm charts on are also generated.
Furthermore, it filters any branches which are not used for releases, to avoid generating wrong files.

This task finds the latest release branch using the branches from the GitHub repository of the DTO.
First, the repository is cloned.
Then, the command \\
\verb%git branch -a | grep "${RELEASE_BRANCH_PREFIX}" > ../branches% \\
is used to query all branches, filter the ones with the appropriate prefix and write their names into a file.
The prefix which is used for filtering the names can be configured with the \verb|release_branch_prefix| parameter.
Afterwards a python script, which takes the following parameters, is called.

\begin{table}
    \centering
    \caption{Parameters for the python script finding a release branch}
    \label{tab:py-finding-the-release-branch}
    \begin{tabular}{l|l}
        Parameter & Expected value \\
        \hline
        \verb|--branch-list| & The path to the file containing the filtered branch names \\
        \verb|--release-prefix| & The prefix used to mark release branches. \\
        & It can be configured using the \verb|release_branch_prefix| parameter. \\
        \verb|--csv-prefix| & A prefix for a branch on which the CSV files are generated. \\
        & It can be configured using the \verb|csv_branch_prefix| parameter. \\
        \verb|--helm-prefix| & A prefix for a branch on which the Helm charts are generated. \\
        & It can be configured using the \verb|helm_branch_prefix| parameter. \\
        \verb|--output| & The path to which the results are written. \\
    \end{tabular}
\end{table}

This python script, using the parameters shown in table \ref{tab:py-finding-the-release-branch}, reads through the list of branches, tries to find a version for each one and then compares the versions found.
It does so, by first removing the strings `{release-prefix}-v` and `{release-prefix}-` from the name.
The resulting name is then split by a dot, i.e., `.`, because it assumes the usage of a semantic versioning scheme.
Since the branches usually contain two version numbers, the major and minor version, the resulting array is extended with `.0` until is length is two.
These steps are necessary to obtain a normalized array for further processing.

The normalized arrays of two branches are then compared to each other.
If an array has a length greater than two, it is not in the expected form for release branches and is discarded.
If an element of the array cannot be parsed to an integer value, it is not in the semver format and is discarded.
If the elements of both arrays can be parsed, they are compared to each other.
If one element is larger then the element with the same index in the other array, it is considered as the later version.

Examples:
\begin{itemize}
    \item The branch `release-v0.2` is considered lower than `release-v0.3`
    \item The branch `release-0.2`  is considered lower than `release-v0.2.1`
    \item The branch `release-1.3` is considered later than `release-0.5`
    \item The branch `release-2.0.0` is considered invalid since is has too many version numbers. \\ Therefore any other branch ''wins'' the comparison.
    \item The branch `release-vA.B` is considered invalid since `A` cannot be parsed as an integer. \\ Therefore any other branch ''wins'' the comparison.
    \item The branch `release-1` is considered later than `release-0.9`. \\ The former version is appended with zeros. \\ The actual comparison would be `1.0` against `0.9`.
\end{itemize}

After the latest release branch is found, the version from it is also used to generate the CSV and Helm chart branches.
The resulting branches are \verb`{csv-prefix}-{version}` and \verb`{helm-prefix}-{version}`.
The following string is then written to the \verb`{output}` path.

\begin{verbatim}
{
    "branch": "<latest release branch>",
    "version": "<version of the latest release branch>",
    "csv_branch": "<generated CSV branch name>",
    "helm_branch" "<generated Helm branch name>"
}
\end{verbatim}


\pagebreak

\subsection{Get next version}\label{subsec:get-next-version}

This task uses the version found in the previous task from section \ref{subsec:finding-the-latest-release-branch} to infer the next patch version.
It does so by first querying all GitHub releases.
Then, they are filtered by the version found in the previous task.
For example, if releases exist on GitHub for the version 0.2.0, 0.2.1, 0.3.0, 0.3.1, and 0.4.0, and the release branch has a 0.4 postfix, only the 0.4.0 release is considered.
After the releases have been filtered, their titles, consisting of the version, are compared, similar to the comparison done between branches in the previous task.
From the latest release, the patch version is increase by one, and then saved in the JSON result as the new version.

The purpose of this task is to find the next version number.
Since the release branch names do not contain a patch number, it has to be inferred.
If releases exist for a \verb|major.minor| combination, the patch version of the latest release can just be bumped.
Otherwise, the patch version is \verb|0|.

The querying, filtering and comparison of releases is done again using a python script.
For the script to work, a GitHub token and the target repository have to be supplied as well as the branch version.
Therefore, the script takes the arguments shown in table \ref{tab:py-finding-the-next-version}.

\begin{table}[H]
    \centering
    \caption{Parameters for the python script finding the next version}
    \label{tab:py-finding-the-next-version}
    \begin{tabular}{p{0.3\linewidth}|p{0.6\linewidth}}
        Parameter & Expected value \\
        \hline
        \verb|--version| & Expects the version found by parsing the release branches from the previous task.  \\
        \verb|--token| &  A personal access token for GitHub.
            It should have the following scopes \verb|admin:org, delete_repo, repo, workflow|.
            The same token can be used by multiple tasks if these scopes are set. \\
        \verb|--operator-repository| & The target repository from which the releases are queried. \\
        \verb|--output| & The path to which the calculated version is written. \\
    \end{tabular}
\end{table}

After parsing the arguments, the script creates an instance of \verb|GitHubApi|.
This class implements the Representational State Transfer (REST) requests provided by the GitHub API.
From this instance, the \verb|find_releases| function is called.
The URL used for the request is built by a static prefix \verb|https://api.github.com/repos/|, followed by the target repository name.
A \verb|/releases| prefix is then added as well to specifically target releases.
E.g., if the repository's name is \verb|Dynatrace/dynatrace-operator| the resulting URL is \verb|https://api.github.com/repos/Dynatrace/dynatrace-operator/releases|.
The header for this request contains the fields shown in table \ref{tab:http-header}.

\begin{table}[H]
    \centering
    \caption{HTTP Header}
    \label{tab:http-header}
    \begin{tabular}{p{0.3\linewidth}|p{0.6\linewidth}}
        Header field name & Value \\
        \hline
        Accept & \verb|application/vnd.github.v3+json| \\
        Authorization & \verb|token <GitHub Token>| \\
        Content-Type & \verb|application/json| \\
    \end{tabular}
\end{table}

If the request does not succeed, an error is returned.
Otherwise, the body of the request contains all releases for the target repository as a JSON-array.
The body is then converted into an array of instance of type \verb|GitHubRelease|.
This class does not implement specific functions, but is only used to hold the data from the request.

All found releases are then filtered.
Releases are discarded if:
\begin{itemize}
    \item They are a draft
    \item They do not contain the release branch version
    \item They are a sub-release of some kind.
        I.e., if they do not contain the character ``-''
\end{itemize}

Every release that was not discarded is then compared against the other, not discarded, releases.
The latest one is then taken.
From this release, the version is parsed from the tag and the patch version is increased by one.
If there exists no release for the current major-minor version combination, the resulting version defaults to \verb|major.minor.0|.
The result is then written to the output path.
Finally, the task finishes after updating the resulting JSON-data with the next version.


\pagebreak

\subsection{Get release information}\label{subsec:get-release-information}

This task is used to generate a changelog and collect metadata.
First, the branches SHA-hash is determined, by cloning the target repository and switching to the release branch, determined by the [find-latest-release-branch](../release%20tasks/find-latest-release-branch.md) task.
The bash command `git rev-parse "${branch}"` then returns the hash value, which is used later.

Then, the prefix for the Google Container Registry (GCR) is determined.
The task uses the GKE service account, read from a vault secret, to query the project name.
By adding the prefix `gcr.io/` and replacing colons with forward slashes, the registry for the image, which is built and uploaded in a later task, can be determined.
The project name depends on the marketplace that is set to be used.
It can be configured by changing the `gcp_marketplace` parameter in the [params-dev](../../params-dev.yaml) or [params-prod](../../params-prod.yaml) files.
Usually, these steps result in the prefix `gcr.io/{gcp_marketplace}`.

After the project name and the branch hash are determined, a python script is called.
It can be found in [/scripts/release/metadata/](../../scripts/release/metadata/__init__.py).
This script takes the following parameters.

| Parameter               | Expected value                                                                                                                                                                         | Default                        |
| ----------------------- | -------------------------------------------------------------------------------------------------------------------------------------------------------------------------------------- | ------------------------------ |
| `--release-data`        | The path to the release data currently compiled by previous tasks.                                                                                                                     | `release-data/version`         |
| `--branch-sha`          | The release branches hash value as determined before.                                                                                                                                  |                                |
| `--github-token`        | A personal access token for GitHub. It should have the following scopes `admin:org, delete_repo, repo, workflow`. The same token can be used by multiple tasks if these scopes are set |                                |
| `--operator-repository` | The target repository from which the releases are queried                                                                                                                              | `Dynatrace/dynatrace-operator` |
| `--gke-registry`        | The name of the GCR registry for the GKE release step later on.                                                                                                                        |                                |
| `--gke-app-name`        | The name under which the DTO is released.                                                                                                                                              | `dynatrace-operator`           |

This script then proceeds to compile the given data, excluding the token and repository, together into a dictionary.
The descriptions for the GitHub and Helm releases are also added.
Both are hardcoded and can be found at [/scripts/release/metadata/descriptions.py](/../../scripts/release/metadata/descriptions.py).
It then uses the token to request a changelog for the given repository from the GitHub API.
Similarly to how [get-next-version](./get-next-version.md) requests releases, this script calls the `https://api.github.com/repos/{operator_repository}/releases/generate-notes` endpoint.
The generated changelog is then also appended to the resulting release data.

This task should then output the following JSON document.
```json
    {
    "branch": "<release-branch>",
    "version": "<next version to be release>",
    "csv_branch": "<branch on which to generate CSV files>",
    "helm_branch": "<branch on which to generate the Helm release>",
    "tag": "<the tag for the GitHub release>",
    "sha": "<hash of the release branch>",
    "release": {
        "changelog": "<Changelog generated by the GitHub API>",
        "description": "<Hardcoded description for the GitHub release>",
        "helm-chart-description": "<Hardcoded description for the Helm release>"
    },
    "gcr": {
        "registry": "<GCR registry as determined before>",
        "app-name": "dynatrace-operator",
        "version": "<next version to be release>",
        "image": "<GCR image name for the operator release>"
    }
}
```

## Purpose

This tasks main purpose is to get the generated changelogs from the GitHub API.
To do so, the GitHub API expects the branch hash to now for which commit to generate changelogs for.
Other information which is conveniently available is then also added to the resulting release data.


\textbf{Image preflight}

The purpose of the task is to automatically check if necessary images are available on all registries.
Furthermore, it finds the digests of the DockerHub and RHCC images, so the images in the CSVs can be digest-pinned.

This task is used to check whether all images necessary for this release are available.
It needs the Docker config defined as a vault secret to have access to the necessary registries.
Furthermore, it uses a Docker in Docker image to try and pull the images.
Finally, it parses the digests of the images and adds them to the release information.

First \verb|jq| is installed to read the release data, because \verb|jq| is not installed by default on the Docker in Docker image.
Then, the Docker config provided is written to \verb|~/.docker/config| \verb|.json| to make it available to Docker.
The command \verb|source /docker-lib.sh| imports all functions made available by the Docker in Docker image.
One of the functions is \verb|start_docker|, which is then called to start the Docker daemon.

Then the availability of the following images is checked.
\begin{itemize}
    \item \verb|registry.connect.redhat.com/dynatrace/dynatrace-operator:v${version}|
    \item \verb|gcr.io/${GCE_MARKETPLACE}/dynatrace-operator:${version}|
    \item \verb|docker.io/dynatrace/dynatrace-operator:v${version}|
\end{itemize}

The value for \verb|version| is read from the release data using jq while \verb|${GCE_MARKETPLACE}| is defined by the \verb|gcp_marketplace| parameter.
If any of those image cannot be pulled, the task fails.

The digest is then parsed for each image.
From the \verb|docker pull| command, for the DockerHub and RHCC image, the response is taken.
Since the digest can be found in the response, it is parsed using \verb|grep "Digest:"| to find the line.
Then, using shell expansion, the ``Digest: '' prefix is removed with \verb|${digest#"Digest: "}|.
This functionality is used twice and therefore put into a function called \verb|parse_digest| which takes the response from the pull command as its parameter.


\textbf{Prepare files for release}

The task generates the CSV bundles for Openshift and Kubernetes OLM systems.
Since most files and information is available, minor changes like setting version properties are also done.

This task is used to generate the CSV bundles for the OLM releases as well as setting minor details in other files.
To do so, it first needs to install the Operator SDK.
The version to download can be configured with the \verb|operator_sdk_version| parameter.
If the parameter is not set, the latest version available is downloaded.

Then the release branch is checked out of the target repository.
From this branch, the CSV branch is created.
In order to get a clean branch, it is first deleted if it exists and the deletion is force pushed afterwards.
Force deleting the branch will close open pull requests that are based on this branch.

Afterwards, the following changes are made. \\
In \verb|<repository>/config/helm/chart/default/Chart.yaml|:
\begin{itemize}
    \item The \verb| version | property is set to the version which is going to be released
    \item The \verb| appVersion | property is set to the version which is going to be released
    \item Setting this property is done using a library, which does not preserve comments.
    Therefore, the license text is re-added to the top afterwards
\end{itemize}

In \verb|<repository>/config/helm/schema.yaml|
\begin{itemize}
    \item The \verb| x-google-marketplace.publishedVersion | property is set to the version which is going to be released
    \item The \verb| x-google-marketplace.publishedVersionMetadata.releaseNote | property is set to the generated changelog
    \item Again, the license text is re-added afterwards
\end{itemize}

In \verb|dynatrace-operator/config/helm/chart/default/templates/application.yaml|
\begin{itemize}
    \item The \verb|version| property is set to the version which is going to be released
    \item This file is a Helm template, which has some extended notations compared to a normal YAML file.
        Therefore, the same library used before cannot be used here to set this property.
    \item In order to mitigate this problem, the replacement is done using \verb|sed| and the license text does not need to be re-added.
\end{itemize}

Then, the CSV bundles are generated.
Once for Kubernetes, another time for Openshift.
A target of the Makefile included in the dynatrace-operator repository is used to generate these files.
The following parameters are set to make sure the image is correctly replaced for each platform.

For Kubernetes, the following parameters are set.

\begin{verbatim}
IMG="docker.io/dynatrace/dynatrace-operator:v${version}"
MASTER_IMAGE="docker.io/dynatrace/dynatrace-operator:v${version}"
BRANCH_IMAGE="docker.io/dynatrace/dynatrace-operator:v${version}"
OLM_IMAGE="docker.io/dynatrace/dynatrace-operator:v${version}"
\end{verbatim}

For Openshift, the following parameters are set.

\begin{verbatim}
IMG="registry.connect.redhat.com/dynatrace/dynatrace-operator:v${version}"
BRANCH_IMAGE="registry.connect.redhat.com/dynatrace/dynatrace-operator:v${version}"
MASTER_IMAGE="registry.connect.redhat.com/dynatrace/dynatrace-operator:v${version}"
OLM_IMAGE="registry.connect.redhat.com/dynatrace/dynatrace-operator:v${version}"
\end{verbatim}

Finally, the generated CSV files are finalized by a python script.
This script main purpose is to set the correct date for the \verb|metadata/annotations/createdAt| property in the main CSV files.
It also sets the \verb|spec/replaces| field to the version of the last GitHub release, however, this value is only used as a fallback.
This property is later overwritten to depend on the release CSV files instead of the released GitHub versions.

After all changes were made, they are pushed to the CSV branch.
Finally, a pull request from the CSV branch to the release branch is created.



\subsection{Release GKE}\label{subsec:release-gke}

This job consists of multiple tasks.
Their goal is to build and push the deployer image for GCM, prepare a GKE cluster, install the deployer image on it and check if a basic Operator deployment succeeds.

## build-and-publish-deployer-image

This task builds the Dockerfile at `<repository>/config/helm/` and pushes the result as the deployer image.

### Purpose

The purpose of this task is to build a deployer image and publish it to GCR.

## prepare-cluster

This task prepares a GKE cluster to be able to deploy the previously built deployer image.
First, the kube-config is read from the vault secret, which is set by [/tasks/20_fetch-kubeconfig/](../../tasks/20_fetch-kubeconfig/task.sh).
Therefore, for this task, and by extension this job, to succeed, a GKE cluster must be running.
A GKE cluster can be created with [/tasks/10_create-infrastructure/](../../tasks/10_create-infrastructure/task.sh), after which `fetch-kubeconfig` must be run.

After the kube-config is read and written to `\${HOME}/.kube/config`, the GCP application CRD is applied from `https://raw.githubusercontent.com/GoogleCloudPlatform/marketplace-k8s-app-tools/master/crd/app-crd.yaml`.
This CRD is needed so GCP applications can be deployed.
If the `dynatrace` namespace or the DynaKube CRD exist on the cluster, they are deleted to ensure a clean state.
Afterwards, the namespace is created and the DynaKube CRD is applied again.

### Purpose

The purpose of this task is to prepare a GKE cluster for the deployment of deployer image.
This includes cleaning any leftover resources created by a previous DTO deployment and applying the GKE Application CRD.

### Configuration options

The parameters can be changed in the [params-dev](../../params-dev.yaml) or [params-prod](../../params-prod.yaml) files.
The following parameters can be used to configure this task.

| Parameter             | Effect                                                    | Production default             | Development default                     |
| --------------------- | --------------------------------------------------------- | ------------------------------ | --------------------------------------- |
| `operator_repository` | Defines in which repository the Dockerfile is looked for. | `Dynatrace/dynatrace-operator` | `dt-team-kubernetes/dynatrace-operator` |

### Common error cases

During development, no common error cases were found.
However, if this task fails, check if the kube-config secret exists and the content points to a running GKE cluster.

## install-deployer-image

This task uses Google's `mpdev` tool to deploy the deployer image onto the cluster mentioned above.
Since it also uses a Docker in Docker image, it first install `jq` to read the image name from the release data.
It then installs the `mpdev` tool and prepares the kube-config file as well as the GKE service account.

The `mpdev` tool can take parameters which are passed to the custom resource of the DTO.
In order to check a simple deployment, the following properties are given to the tool in a JSON format.

```json
    {
    "name": "dynatrace-operator",
    "namespace": "dynatrace",
    "apiUrl": "<a valid API url>",
    "apiToken": "<a valid api token for the given API>",
    "paasToken": "<a valid paas token for the given API>"
}
```

The API URL, the API token, as well as the PAAS token, are defined as vault secrets.
After executing the command `./mpdev install --deployer="${image}" --parameters="${parameters}"`, `mpdev` deploys the deployer image with the parameters from above.

### Purpose

This tasks purpose is to deploy the deployer image on a prepared GKE cluster.
The deployment is then monitored by the next task.

## sanity-check-deployer

This task checks the deployment job the `mpdev` tool creates for its success.
To do so, it again sets up the kube-config.
Afterwards, the condition type of the job `job/dynatrace-operator-deployer` is monitored.
If it does not reach the state `Complete` after five minutes, the task fails.

If the job reaches this state, the `.status.succeeded` flag is checked.
If the job did not succeed, the task fails.
If the job succeeded, the pod states are checked by [/scripts/check_pod_status.py](../../scripts/check_pod_status.py).
If the pod's states are `Terminated` and the reason is `Error`, the task fails.

If the job completes successfully and all DTO pods are running afterwards, the task succeeds and the deployer image can be assumed to work.

### Purpose

This tasks purpose is to make a quick test whether the deployer image can deploy the DTO successfully.


\subsection{Release GitHub}\label{subsec:release-github}

This task creates a GitHub release for the manifest files and the DTO Helm charts.
It first checks out the release branch of the DTO repository and creates a new branch with the value of `helm_branch` in the release data.
If this branch already exists it is force deleted, to also close and remove any open pull requests.
This way, a consistent state is ensured.

Then, the Helm package is generated.

First, the GPG keyring, which is defined as a vault secret, is setup.
A directory is created at `~/.gnupg/`.
The secret for `pubring` is decoded and written to `~/.gnupg/pubring.gpg`.
The secret for `secring` is decoded and written to `~/.gnupg/secring.gpg`.
The secret for the password to the keyring is written to `~/.gnupg/password`.

The folder `~/.gnupg/` is then given the `700` permission bits.
The created files get the `600` permission bits.
By issuing the command `gpg --list-secret-keys`, the existence of the keyring can be confirmed.

Then, the following command is issued.

```
helm package \
"./dynatrace-operator/config/helm/chart/default/" \
-d "./dynatrace-operator/config/helm/repos/stable/" \
--app-version "${version}" \
--version "${version}" \
--sign \
--key "Dynatrace LLC" \
--keyring ~/.gnupg/secring.gpg \
--passphrase-file ~/.gnupg/password
```

The first argument points to the Helm charts which are packaged.
The `-d` option defines where the new package is written to.
With the `--sign` option, the GPG keyring is used to sign the package, while the `--keyring` option defines which keyring to use.
`--key` sets which of the secrets in the keyring to use and `--passphrase-file` defines where to find the password the secret expects.

After the package is generated, the Helm index `index.yaml`, which lists available packages, is copied to `index.yaml.previous`.
The importance of this file is explained later on.
Then, the index is regenerated with the following command.
The `--url` parameter must point to the path where the `index.yaml` will be available after the release.

```
helm repo index ./dynatrace-operator/config/helm/repos/stable/ \
--url "https://raw.githubusercontent.com/\${operator_repository}/master/config/helm/repos/stable"
```

Then a python script, which can be found at [/scripts/prepare_helm_index.py](../../scripts/prepare_helm_index.py), prepares the Helm index.
Here the importance of the copy of the index becomes apparent.
Every entry of a version has a `created` field which stores the date at which it was created.
If the index is regenerated, all those fields are updated with the current date.
An old version would then have the same creation date as the latest one.
Therefore, the correct dates from `index.yaml.previous` are read and then reapplied to the new index using this python script.

Then, the manifests are generated for each platform, Kubernetes and Openshift.
Similar to generating the bundle, i.e. CSV, files, the following properties are added to the `make manifest` command, to make sure the image is set correctly.

Kubernetes:

```
IMG="docker.io/dynatrace/dynatrace-operator:v${version}"
BRANCH_IMAGE="docker.io/dynatrace/dynatrace-operator:v${version}"
MASTER_IMAGE="docker.io/dynatrace/dynatrace-operator:v${version}"
OLM_IMAGE="docker.io/dynatrace/dynatrace-operator:v${version}"
```

Openshift:

```
IMG="registry.connect.redhat.com/dynatrace/dynatrace-operator:v${version}"
BRANCH_IMAGE="registry.connect.redhat.com/dynatrace/dynatrace-operator:v${version}"
MASTER_IMAGE="registry.connect.redhat.com/dynatrace/dynatrace-operator:v${version}"
OLM_IMAGE="registry.connect.redhat.com/dynatrace/dynatrace-operator:v${version}"
```

The generated Helm package, index and manifests are then added, committed and pushed to the repository.
A pull request is then created.

A python script, found in [/scripts/release/github/](../../scripts/release/github/__init__.py), then creates a draft release and uploads the artifacts to it.
For the script to work, a GitHub token and the target repository have to be supplied as well as the generated release data.
Therefore, the script takes the following arguments.

| Parameter               | Expected value                                                                                                                                                                         | Default                        |
| ----------------------- | -------------------------------------------------------------------------------------------------------------------------------------------------------------------------------------- | ------------------------------ |
| `--release-data`        | Expects the version found by parsing the release branches form the previous task                                                                                                       |                                |
| `--token`               | A personal access token for GitHub. It should have the following scopes `admin:org, delete_repo, repo, workflow`. The same token can be used by multiple tasks if these scopes are set |                                |
| `--operator-repository` | The target repository from which the releases are queried                                                                                                                              | `Dynatrace/dynatrace-operator` |


After parsing the arguments, the script creates an instance of `GitHubApi`.
A class found in [/scripts/release/github/github_api.py](../../scripts/release/github/github_api.py).
This class implements some Representational State Transfer (REST) requests provided by the GitHub API.

From the `GitHubApi` instance, the `create_release` function is called.
This function takes an instance of `GitHubRelease`, which can be found [/scripts/release/github/github_release.py](../../scripts/release/github/github_release.py).
`GitHubRelease` bundles all information needed for a release.
Hardcoded documentation text can be found in [/scripts/release/metadata/descriptions.py](../../scripts/release/metadata/descriptions.py)

The `create_release` function then uses the release data to send a post request to GitHub's release endpoint.
The URL used for the request is built by a static prefix `https://api.github.com/repos/`, followed by the target repository name.
A `/releases` prefix is then added as well to specifically target releases.
E.g., if the repository's name is `Dynatrace/dynatrace-operator` the resulting URL is `https://api.github.com/repos/Dynatrace/dynatrace-operator/releases`.

The header for this request contains the following fields.
| Header field name | Value                            |
| ----------------- | -------------------------------- |
| Accept            | `application/vnd.github.v3+json` |
| Authorization     | `token <GitHub Token>`           |
| Content-Type      | `application/json`               |

The changelog for the GitHub release is left to be generated by GitHub.
If the request does not succeed and does not return a response with a 201 HTTP code, an error is printed and nothing is returned.
The response body of a successful request contains a URL to upload assets to, the `upload_url`.
This upload URL is returned by the `create_release` function.

The script then proceeds to upload the following assets, which were previously generated.
```
dynatrace-operator/config/crd/bases/dynatrace.com_dynakubes.yaml
dynatrace-operator/config/deploy/openshift/openshift.yaml
dynatrace-operator/config/deploy/openshift/openshift-csi.yaml
dynatrace-operator/config/deploy/kubernetes/kubernetes.yaml
dynatrace-operator/config/deploy/kubernetes/kubernetes-csi.yaml
dynatrace-operator/config/helm/repos/stable/dynatrace-operator-{release_data['version']}.tgz
dynatrace-operator/config/helm/repos/stable/dynatrace-operator-{release_data['version']}.tgz.prov
```

In order to upload assets, the `try_upload_asset` function of the `GitHubApi` instance is used.
It uploads the assets to the `upload_url` from before.
The header for the request contains the following fields.

| Header field name | Value                                                                    |
| ----------------- | ------------------------------------------------------------------------ |
| Accept            | `application/vnd.github.v3+json`                                         |
| Authorization     | `token <GitHub Token>`                                                   |
| Content-Type      | Depending on the file's mimetype. Defaults to `application/octet-stream` |

If the upload fails, `try_upload_asset` returns `False` and prints the error message, otherwise it returns true `True`.
If uploading an asset fails, the task fails.

## Purpose

This tasks main goal is to create a release draft on GitHub.
Additionally, it automatically uploads the manifests and Helm chart packages generated by previous steps



\subsection{Release Rancher}\label{subsec:release-rancher}

This task releases the DTO helm chart to the Rancher repository for partner charts.
It first clones the DTO repository and then create a fork of Rancher's repository.
Forking is done by a python script that can be found at [/scripts/release/fork.py](../../scripts/release/fork.py).
This script takes the following arguments.

| Parameter               | Expected value                                                                                                                                                                         | Default          |
| ----------------------- | -------------------------------------------------------------------------------------------------------------------------------------------------------------------------------------- | ---------------- |
| `--token`               | A personal access token for GitHub. It should have the following scopes `admin:org, delete_repo, repo, workflow`. The same token can be used by multiple tasks if these scopes are set |                  |
| `--original-repository` | The original repository which should be forked                                                                                                                                         |                  |
| `--forked-repository`   | The name the forked repository should have                                                                                                                                             |                  |
| `--output-file`         | The path to which the metadata of the fork is written to                                                                                                                               | `fork-data.json` |
| `--retries`             | How often the scripts should retry requests should they fail                                                                                                                           | `10`             |
| `--retry-timeout`       | How long, in seconds, the script waits between request retries.                                                                                                                        | `5`              |

After parsing the arguments, an instance of `GitHubApi`, a class found in [/scripts/release/github/github_api.py](../../scripts/release/github/github_api.py), is created using `{forked-repository}` and `{token}`.
This class implements some REST requests GitHub's API provides.
First, the `exists` function is called, which returns `True` if the forked repository already exists.
To ensure a consistent environment, forked repositories are deleted if they already exist.
Therefore, if the `exist` function returns true, the fork is deleted by calling `delete_repository`, which returns `True` if the request succeeds or `False` if it doesnt.
Additionally, the response from the GitHub API is also returned.

If the forked repository does not exist or has successfully been deleted, it is recreated.
This is done by calling the `fork` function and passing `{original-repository}` as a parameters.
On a success, a check is made, if the forked repository has the expected name `{forked-repository}`.
If it does not, the repository is renamed by instantiating `GitHubApi` with the newly forked repository and calling the `rename` function.
The metadata of the forked repository is then written to `{output-file}`.

When the fork has been created, the task continues to clone it and check out a new branch `dynatrace-operator-v${version}`.
On this branch, a new directory for the version of the new release is created, committed and pushed.
Then, the contents of `{repository}/config/helm/chart/default/`, i.e., the DTO Helm charts, are copied to this new directory.
In the copy of the `Chart.yaml`, the icon property is then set to `file://../logo.png`, to reference the local file.
Finally, the changes are add, committed and pushed to the forked repository.

## Purpose

The purpose of this task is to execute all steps necessary to release the DTO Helm charts to Rancher.
This includes forking their repository and follow their guidelines.
A pull request or draft pull request is not done to avoid operating on a foreign repository with automatic tasks.


\subsection{Release community operators}\label{subsec:release-community-operators}

This task releases the CSV bundle for Kubernetes to the community operators repository.
Releasing the CSV files to this repository makes the DTO available on OperatorHub.
The purpose of this task is to execute all steps necessary to release the CSV bundle to OperatorHub.
This includes forking the repository and following the OperatorHub guidelines.
A pull request or draft pull request is not done to avoid operating on a foreign repository with automatic tasks.

It first clones the DTO repository and then creates a fork of the community operators repository using a Python script
This script takes the following arguments.

\begin{table}
    \centering
    \caption{Parameters to clone and fork a repository}
    \label{tab:parameters-to-clone-and-fork-a-repository}
    \begin{tabular}{|p{0.3\linewidth}|p{0.7\linewidth}|}
        Parameter & Expected value \\
        \hline
        \verb|--token| & A personal access token for GitHub.
            It should have the following scopes \verb|admin:org, delete_repo, repo, workflow|.
            The same token can be used by multiple tasks if these scopes are set. \\
        \verb|--original-repository| & The original repository which should be forked. \\
        \verb|--forked-repository| & The name the forked repository should have. \\
        \verb|--output-file| & The path to which the metadata of the fork is written to. \\
        \verb|--retries| & How often the scripts should retry requests should they fail. \\
        \verb|--retry-timeout| & How long, in seconds, the script waits between request retries. \\
    \end{tabular}
\end{table}

After parsing the arguments, an instance of \verb|GitHubApi| is created using \verb|{forked-repository}| and \verb|{token}|.
This class implements some REST requests GitHub's API provides.
First, the \verb|exists| function is called, which returns \verb|True| if the forked repository already exists.
To ensure a consistent environment, forked repositories are deleted if they already exist.
Therefore, if the \verb|exist| function returns true, the fork is deleted by calling \verb|delete_repository|, which returns \verb|True| if the request succeeds or \verb|False| if it doesnt.
Additionally, the response from the GitHub API is also returned.

If the forked repository does not exist or has successfully been deleted, it is recreated.
This is done by calling the \verb|fork| function and passing \verb|{original-repository}| as a parameters.
On a success, a check is made, if the forked repository has the expected name \verb|{forked-repository}|.
If it does not, the repository is renamed by instantiating \verb|GitHubApi| with the newly forked repository and calling the \verb|rename| function.
The metadata of the forked repository is then written to \verb|{output-file}|.
In this forked repository, the following folders are created for the new version.

\begin{verbatim}
${fork_name}/operators/dynatrace-operator/${version}/manifests
${fork_name}/operators/dynatrace-operator/${version}/metadata
\end{verbatim}

The Dockerfile located at \verb|{repository}/config/olm/kubernetes/bundle-${version}.Dockerfile| is placed in the version's folder.
The previously mentioned folders, are then populated with the CSV bundles from \verb|{repository}/config/olm/kubernetes/${version}/manifests| and \verb|{repository}/config/olm/kubernetes/${version}/metadata| respectively.
Finally, the changes are added, committed and pushed to the forked repository.


\subsection{Release RedHat marketplace}\label{subsec:release-redhat-marketplace}

This task releases the CSV bundle for Openshift to the RedHat marketplace repository.
Releasing the CSV files to this repository makes the DTO available on the RedHat Marketplace for Operators.
The purpose of this task is to execute all steps necessary to release the CSV bundle to the RedHat Marketplace for Operators.
This includes forking the repository and following the RedHat Marketplace guidelines.
A pull request or draft pull request is not done to avoid operating on a foreign repository with automatic tasks.

It first clones the DTO repository and then creates a fork of the RedHat marketplace repository.
Forking is done by a python script which takes the arguments shown in table\ \ref{tab:parameters-to-fork-the-redhat-marketplace-repository}.

\begin{table}[H]
    \centering
    \caption{Parameters to fork the RedHat marketplace repository}
    \label{tab:parameters-to-fork-the-redhat-marketplace-repository}
    \begin{tabular}{p{0.3\linewidth}|p{0.6\linewidth}}
        Parameter & Expected value \\
        \hline
        \verb|--token| & A personal access token for GitHub.
            It should have the following scopes \verb|admin:org, delete_repo, repo, workflow|.
            The same token can be used by multiple tasks if these scopes are set. \\
        \verb|--original-repository| & The original repository which should be fork. \\
        \verb|--forked-repository| & The name the forked repository should have. \\
        \verb|--output-file| & The path to which the metadata of the fork is written to. \\
        \verb|--retries| & How often the scripts should retry requests should they fail. \\
        \verb|--retry-timeout| & How long, in seconds, the script waits between request retries. \\
    \end{tabular}
\end{table}

After parsing the arguments, an instance of \verb|GitHubApi| is created using \verb|{forked-repository}| and \verb|{token}|.
This class implements some REST requests GitHub's API provides.
First, the \verb|exists| function is called, which returns \verb|True| if the forked repository already exists.

\pagebreak

To ensure a consistent environment, forked repositories are deleted if they already exist.
Therefore, if the \verb|exist| function returns true, the fork is deleted by calling \verb|delete_repository|, which returns \verb|True| if the request succeeds or \verb|False| if it doesn't.
Additionally, the response from the GitHub API is also returned.

If the forked repository does not exist or has successfully been deleted, it is recreated.
This is done by calling the \verb|fork| function and passing \verb|{original-repository}| as a parameters.
On a success, a check is made, if the forked repository has the expected name \verb|{forked-repository}|.
If it does not, the repository is renamed by instantiating \verb|GitHubApi| with the newly forked repository and calling the \verb|rename| function.
The metadata of the forked repository is then written to \verb|{output-file}|.

In this forked repository, first, the latest released version is determined.
A python script iterates over the directories in the \verb|dynatrace-operator-rhmp| folder.
The directories are named after the released version, i.e., \verb|0.2.2|, \verb|0.3.0|, etc.
The latest version is then saved in a variable \verb|last_version|.
If no version exist, it defaults to an empty string.

Afterwards, a new folder for the version that is to be released is created.
The folders for the manifest and metadata are then populated with the generated CSV bundles respectively.
In the CSV file of the version to be released, the \verb|replaces| property is then updated, if \verb|last_version| is not empty.
If the variable is empty, the property is removed.
Setting this property allows Openshift to correctly update the DTO.

Then, annotations mentioning the string ``dynatrace-operator'' are changed to read ``dynatrace-operator-rhmp'' and the CSV file is renamed from  mentioning ``dynatrace-operator'' to mentioning ``dynatrace-operator-rhmp'' as well.
This is necessary to reflect the different directory name RedHat has given its Marketplace releases.
Finally, the changes are added, committed and pushed to the forked repository.


\subsection{release-certified-operators}\label{subsec:release-certified-operators}

This task releases the CSV bundle for Openshift to the RedHat certified operators repository.
Releasing the CSV files to this repository makes the DTO available on the RedHat Catalog for certified Operators.

It first clones the DTO repository and then creates a fork of the RedHat certified operators repository.
Forking is done by a python script that can be found at [/scripts/release/fork.py](../../scripts/release/fork.py).
This script takes the following arguments.

| Parameter               | Expected value                                                                                                                                                                         | Default          |
| ----------------------- | -------------------------------------------------------------------------------------------------------------------------------------------------------------------------------------- | ---------------- |
| `--token`               | A personal access token for GitHub. It should have the following scopes `admin:org, delete_repo, repo, workflow`. The same token can be used by multiple tasks if these scopes are set |                  |
| `--original-repository` | The original repository which should be forked                                                                                                                                         |                  |
| `--forked-repository`   | The name the forked repository should have                                                                                                                                             |                  |
| `--output-file`         | The path to which the metadata of the fork is written to                                                                                                                               | `fork-data.json` |
| `--retries`             | How often the scripts should retry requests should they fail                                                                                                                           | `10`             |
| `--retry-timeout`       | How long, in seconds, the script waits between request retries.                                                                                                                        | `5`              |

After parsing the arguments, an instance of `GitHubApi`, a class found in [/scripts/release/github/github_api.py](../../scripts/release/github/github_api.py), is created using `{forked-repository}` and `{token}`.
This class implements some REST requests GitHub's API provides.
First, the `exists` function is called, which returns `True` if the forked repository already exists.
To ensure a consistent environment, forked repositories are deleted if they already exist.
Therefore, if the `exist` function returns true, the fork is deleted by calling `delete_repository`, which returns `True` if the request succeeds or `False` if it doesnt.
Additionally, the response from the GitHub API is also returned.

If the forked repository does not exist or has successfully been deleted, it is recreated.
This is done by calling the `fork` function and passing `{original-repository}` as a parameters.
On a success, a check is made, if the forked repository has the expected name `{forked-repository}`.
If it does not, the repository is renamed by instantiating `GitHubApi` with the newly forked repository and calling the `rename` function.
The metadata of the forked repository is then written to `{output-file}`.

In this forked repository, first, the latest released version is determined.
A python script, located at [/scripts/release/directory_version.py](../../scripts/release/directory_version.py), iterates over the directories in the `dynatrace-operator` folder.
The directories are named after the released version, i.e., `0.2.2`, `0.3.0`, etc.
The latest version is then saved in a variable `last_version`.
If no version exist, it defaults to an empty string.

Afterwards, a new folder for the version to be released is created.
The folders, `${fork_name}/operators/dynatrace-operator/${version}/manifests` and `${fork_name}/operators/dynatrace-operator/${version}/metadata` are then populated with the CSV bundles from `{repository}/config/olm/openshift/${version}/manifests` and `{repository}/config/olm/openshift/${version}/metadata` respectively.
In the CSV file of the version to be released, the `replaces` property is then updated, if `last_version` is not empty.
If the variable is empty, the property is removed.
Setting this property allows Openshift to correctly update the DTO.
Finally, the changes are added, committed and pushed to the forked repository.

\subsubsection{Purpose}\label{subsubsec:release-certified-operators-Purpose}

The purpose of this task is to execute all steps necessary to release the CSV bundle to the RedHat Catalog for Certified Operators.
This includes forking the repository and following the RedHat Catalog guidelines.
A pull request or draft pull request is not done to avoid operating on a foreign repository with automatic tasks.


\textbf{Release community operators prod}

This task releases the CSV bundle for Kubernetes to the community-operators-prod repository.
Releasing the CSV files to this repository makes the DTO available on the embedded OperatorHub in OpenShift and OKD.
The purpose of this task is to execute all steps necessary to release the CSV bundle to the embedded OperatorHub in OpenShift and OKD.
This includes forking the repository and following the OperatorHub guidelines.
A pull request or draft pull request is not done to avoid operating on a foreign repository with automatic tasks.

It first clones the DTO repository and then creates a fork of the community-operators-prod repository.
Forking is done by a python script that takes the arguments shown in table~\ref{tab:parameters-to-fork-the-community-operators-prod-repository}.

\begin{table}[h]
    \centering
    \caption{Parameters to fork the community-operators-prod repository}
    \label{tab:parameters-to-fork-the-community-operators-prod-repository}
    \begin{tabular}{p{0.3\linewidth}|p{0.6\linewidth}}
        Parameter & Expected value \\
        \hline
        \verb|--token| & A personal access token for GitHub.
            It should have the following scopes \verb|admin:org, delete_repo, repo, workflow|.
            The same token can be used by multiple tasks if these scopes are set. \\
        \verb|--original-repository| & The original repository which should be forked. \\
        \verb|--forked-repository| & The name the forked repository should have. \\
        \verb|--output-file| & The path to which the metadata of the fork is written to. \\
        \verb|--retries| & How often the scripts should retry requests should they fail. \\
        \verb|--retry-timeout| & How long, in seconds, the script waits between request retries. \\
    \end{tabular}
\end{table}

After parsing the arguments, an instance of \verb|GitHubApi| is created using \verb|{forked-repository}| and \verb|{token}|.
This class implements some REST requests GitHub's API provides.
First, the \verb|exists| function is called, which returns \verb|True| if the forked repository already exists.
To ensure a consistent environment, forked repositories are deleted if they already exist.
Therefore, if the \verb|exist| function returns true, the fork is deleted by calling \verb|delete_repository|, which returns \verb|True| if the request succeeds or \verb|False| if it doesn't.
Additionally, the response from the GitHub API is also returned.

If the forked repository does not exist or has successfully been deleted, it is recreated.
This is done by calling the \verb|fork| function and passing \verb|{original-repository}| as a parameters.
On a success, a check is made, if the forked repository has the expected name \verb|{forked-repository}|.
If it does not, the repository is renamed by instantiating \verb|GitHubApi| with the newly forked repository and calling the \verb|rename| function.
The metadata of the forked repository is then written to \verb|{output-file}|.

In this forked repository, the folder \verb|version| folder is created inside the folder designated for the DTO for the new version.
The bundle's Dockerfile is placed in the version's folder.
The previously mentioned folder is then also populated with the generated CSV bundles.
Finally, the changes are added, committed and pushed to the forked repository.



\subsection{OpenShift infrastructure}\label{subsec:openshift-infrastructure}

These group of tasks manage the creation and configuration of an OpenShift infrastructure.
This infrastructure can then be used to test the releases for the marketplaces using CSVs and OLM.
In general these tasks do not have to run for every release, but can be used to debug one should errors arise.

# prepare-openshift-infrastructure

This task creates an S3 bucket to place openshift installer state into.
It uses a python script, located at [/scripts/openshift/s3_setup.py](../../scripts/openshift/s3_setup.py), to do so.
This script takes the following parameters.

| Parameter         | Expected value                                                                                                      | Default |
| ----------------- | ------------------------------------------------------------------------------------------------------------------- | ------- |
| `--cluster-name`  | The name of the cluster which will be created. It is usually the same name as the pipeline.                         |         |
| `--bucket-name`   | The name the created bucket should have.                                                                            |         |
| `--bucket-region` | The AWS region the bucket should be created in. Like everything and your doorbell, it is usually set to `us-east-1` |         |
| `--remove`        | A flag to remove an existing bucket again                                                                           | N.A.    |

The `boto3` python library is then used to connect to AWS S3 service.
If the given bucket name does not exist, it is created and an empty file is uploaded as a state file.
If the given bucket name does exist, but a state file is missing, an empty file is created and uploaded.

## Purpose

The purpose of this task it to prepare an S3 bucket for the Openshift installer to place a state file in.

## Configuration options

The parameters can be changed in the [params-dev](../../params-dev.yaml) or [params-prod](../../params-prod.yaml) files.
The following parameters can be used to configure this task.

| Parameter                       | Effect                                                                | Production default                      | Development default                     |
| ------------------------------- | --------------------------------------------------------------------- | --------------------------------------- | --------------------------------------- |
| `cluster_name`                  | Influences the key under which the state file is saved in the bucket. | `operator-release`                      | `operator-release`                      |
| `installer_state_bucket`        | Defines the name under which the bucket is created.                   | `dynatrace-operator-ocp-release-states` | `dynatrace-operator-ocp-release-states` |
| `installer_state_bucket_region` | Defines in which AWS region the bucket is created.                    | `us-east-1`                             | `us-east-1`                             |

## Common error cases

During development, no common error cases were found.


\subsection{Deploy OpenShift cluster}\label{subsec:deploy-openshift-cluster}

This task deploys an Openshift cluster on AWS infrastructure.
First, it checks if a state file exists in the S3 bucket created in the previous task.
The file is then unzipped.
If a install configuration already exists in this file, the installation was already done and the current one is skipped.

Otherwise, Openshift is installed.
First, a new install configuration file is created, based one the \verb|{install_config}| parameter.
In it, the AWS region and the cluster name, which are configured with the \verb|{aws_region}| and \verb|{cluster_name}| respectively, are set.

Downloading the Openshift installer is done in one of two ways.
First, by setting a installer URL.
If this URL is set, the Openshift installer is downloaded from the URL and the script assumes it is a binary.
This way of downloading the Openshift installer is not tested, therefore there is no parameter provided to set the installer URL.

The second way of downloading Openshift, which is supported by the pipeline, is by defining the target version, URL to the Openshift installer directory and whether unstable versions should also be considered or not.
Configuring these values is done with the \verb|{ocp_version}|, \verb|{ocp_installer_directory_url}|, and \verb|{ocp_installer_unstable}| parameters respectively.
The task then downloads the latest patch version of the installer and extracts the downloaded archive containing the installer.
After the installer has been downloaded, it is executed to create the cluster.

When the installer has finished, the deployments VPCs are queried, along with their security groups and NAT gateways.
These VPCs and gateways are then whitelisted AWS's infrastructure.
The installer state, which was written by the installer, is then zipped and uploaded to the S3 bucket.
Finally, the resulting \verb|kubeconfig| is saved as a Vault secret under the name of \verb|openshift|.



# configure-openshift-cluster

This task, like `prepare-openshift-infrastructure` and `deploy-openshift-cluster`, was taken from the repository [pipeline-cpn-kubernetes](https://bitbucket.lab.dynatrace.org/projects/CPN/repos/pipeline-cpn-kubernetes/browse) repository.
It labels nodes and can update the openshift version, if the `{ocp_channel}` has changed.

## Purpose

Minor setup of nodes and updating Openshift.

## Configuration options

The parameters can be changed in the [params-dev](../../params-dev.yaml) or [params-prod](../../params-prod.yaml) files.
The following parameters can be used to configure this task.

| Parameter   | Effect                                                                         | Production default | Development default |  | Production default | Development default |
| ----------- | ------------------------------------------------------------------------------ | ------------------ | ------------------- |
| ocp_channel | Defines which release channel of Openshift should be used to check for update. | `stable`           | `stable`            |

## Common error cases

This task usually only fails if the S3 bucket does not exist. 
I.e., if the previous tasks were not run in the correct order or if `cleanup-openshift-infrastructure` ran in the mean time.
The correct order of tasks is as follows.

1. prepare-openshift-infrastructure
2. deploy-openshift-cluster
3. configure-openshift-cluster
4. configure-openshift-pipeline
5. Optional:
   * test-certified-operators
   * test-redhat-marketplace-operators
6. destroy-openshift-cluster
7. cleanup-openshift-infrastructure


\subsection{configure-openshift-pipeline}\label{subsec:configure-openshift-pipeline}

This tasks executes the steps necessary to deploy RedHat's CI pipeline infrastructure.
These steps are documented in RedHat's [certification-releases](https://github.com/redhat-openshift-ecosystem/certification-releases/blob/main/4.9/ga/ci-pipeline.md#step1) repository.

First, the `kubeconfig` from the Vault secret `openshift` is read and written to `~/kube/config`.
With this config written, a connection to the Openshift cluster can be made.

Then, a function `setup-project` is defined.
This function is executed twice at the bottom of the script.
Once with the argument `release-certified-operators` and another time with `release-redhat-marketplace`.

This function the first defines the given argument as `project_name`.
It then recreates the namespace or project with that name.
The project is first deleted if it exists and then created again.
Since it may take a few seconds for a project to be deleted, the create command is executed until it succeeds.
Then, the new project is set as the currently active one.
The kubeconfig is then applied as a secret to the project.

Afterwards, the RedHat image catalogs `certified-operator-index` and `redhat-marketplace-index` are imported.
Importing those catalogs may take some time, therefore the commands have a timeout of 600 seconds or ten minutes.
It usually does not take ten minutes to import these catalogs, but they are necessary for the steps following, therefor it is set quite high.
The standard output is also redirected to not be included in the task's logs.
This is due to the fact that these commands produce absurd amounts of logs.
If the import needs to be debugged, the lines `    --request-timeout 600 1> /dev/null` can be changed to read `    --request-timeout 600` to include these logs.

Next, the Vault secret `dockerconfig` is created as a pull secret in the project, to allows pulling necessary images.
Then, the GitHub SSH keys are added as a secret to the cluster, to allow cloning the repositories containing CSV files.
I.e., the forked `certified-operators` and `redhat-marketplace-operators` repositories.
Finally, a subscription to the RedHat Certification Pipelines Operator is applied, which can be found at [/scripts/openshift/subscription.yaml](../../scripts/openshift/subscription.yaml).

\subsubsection{Purpose}\label{subsubsec:cop-Purpose}

This tasks purpose is to create the infrastructure necessary to run RedHat's CI pipeline.
This pipeline can then be used to debug and verify the previously generated CSV bundles.


# test-certified-operators

This task runs RedHat's CI pipeline for Operators to check if they can be merged into their [certified-operators](https://github.com/redhat-openshift-ecosystem/certified-operators) repository.
First, the Vault secret `openshift` is read and written to `~/.kube/config` to allow for a connection to the cluster.
Then, `tekton` the CI tool that is needed for RedHat's CI pipeline, is downloaded and installed.
Afterwards, the manifests for the pipeline and its tasks are applied.
Finally, the pipeline is started using `tekton`.

The `tekton` command uses the following parameters.

| Parameters                                                                           | Description                                                                                                                                                                                                                                                                                                                                                                                                                |
| ------------------------------------------------------------------------------------ | -------------------------------------------------------------------------------------------------------------------------------------------------------------------------------------------------------------------------------------------------------------------------------------------------------------------------------------------------------------------------------------------------------------------------- |
| `--use-param-defaults`                                                               | If a parameter is not explicitly given, it uses a default value instead                                                                                                                                                                                                                                                                                                                                                    |
| `--param git_repo_url="${git_repo_url}"`                                             | The URL to the git repository containing the bundle CSV files to be tested                                                                                                                                                                                                                                                                                                                                                 |
| `--param git_branch="${git_branch}"`                                                 | The branch of the aforementioned repository that should be used                                                                                                                                                                                                                                                                                                                                                            |
| `--param bundle_path="${bundle_path}"`                                               | The path inside the repository that points to the bundle to be tested                                                                                                                                                                                                                                                                                                                                                      |
| `--param env=prod`                                                                   | Only used when actively developing RedHat's CI pipeline. I.e., `prod` is always the correct value                                                                                                                                                                                                                                                                                                                          |
| `--param pin_digests=true`                                                           | Defines whether images should be digest pinned. I.e., if the tags of images should be replaced with their SHA hash. Images referenced in a CSV file must be digest pinned to be able to be released to RedHat's marketplaces. If this parameter is set to true, digest pinning is done automatically and pushed to the repository `{git_repo_url}` points to. The branch the changes are pushed to has a `-pinned` suffix. |
| `--param git_username="${GITHUB_USERNAME}"`                                          | Defines the username with which to push the changes from the digest pinning step. The user must have the SSH key, applied in the previous step, configured on their GitHub account                                                                                                                                                                                                                                         |
| `--param git_email="${GITHUB_EMAIL}"`                                                | Defines the email of the user used in `{git_username}`.                                                                                                                                                                                                                                                                                                                                                                    |
| `--workspace name=pipeline,volumeClaimTemplateFile=templates/workspace-template.yml` | This parameter is necessary according to the CI pipelines documentation and is not explained further.                                                                                                                                                                                                                                                                                                                      |
| `--workspace name=kubeconfig,secret=kubeconfig`                                      | Defines how the secret is called under which the kubeconfig has been stored in the namespace. This parameter is necessary according to the CI pipelines documentation and is not explained further.                                                                                                                                                                                                                        |
| `--workspace name=ssh-dir,secret=github-ssh-credentials`                             | Defines how the secret is called under which the GitHub SSH key has been stored in the namespace. This parameters is necessary so the digest-pinned CSV files can be pushed to the repository.                                                                                                                                                                                                                             |
| `--workspace name=registry-credentials,secret=registry-dockerconfig-secret`          | Defines how the secret is called under which the Dockerconfig is stored in the namespace so the digests for images in the CSV files can be queried.                                                                                                                                                                                                                                                                        |
| `--showlog`                                                                          | Enables `tekton` to print the log output of the pipeline tasks to standard output.                                                                                                                                                                                                                                                                                                                                         |

## Important note

Since `tekton` only shows logs of a pipeline tasks which runs in a cluster, it does not exit with an error code if a task fails.
This means, that even if the bundle does not pass the verification, this task is still "green" and marked as successful.
It must be manually checked whether all CI pipeline tasks have succeeded.

## Purpose

This task starts RedHat's CI pipeline for operators and runs it for the certified-operators bundle.
It is used to debug CSV files and OLM deployment, as it is faster and generates more helpful logs as RedHat's hosted CI pipeline.

## Configuration options

There are no parameters in the [params-dev](../../params-dev.yaml) or [params-prod](../../params-prod.yaml) files which influence this task significantly.

## Common error cases

### Task is stuck in `unable to recognize 'ansible/roles/operator-pipeline/templates/openshift/pipelines/operator-ci-pipeline.yml': no matches for kind 'Pipeline' in version 'tekton.dev/v1beta1'`

The RedHat Operator for Pipelines did not create the resource types correctly or has not run at all.
Restart the `configure-openshift-pipeline` task and wait for five minutes.
If this does not solve the problem, destroy and re-deploy the cluster.
If the problem still persists, manually install the operator using the WebUI.
If the problem still persists after that, you are on your own, good luck.

### Task is stuck or fails after `[set-env : set-env]` step

The step after this is cloning the repository.
Therefor, make sure the repository exists and the URL given above is correct.
Also check if the SSH key was correctly applied and is configured in the GitHub account.
Furthermore, check that the branch and bundle path is set correctly.


\input{chapters/implementation/release-pipeline/test-redhat-marketplace-operators}

\textbf{Destroy OpenShift cluster}

This task destroys and deletes an Openshift cluster from AWS infrastructure.
First, it unzips the state file in the S3 bucket created in a previous task.
Then, the Openshift installer is downloaded.
The process of downloading this installer here is the same as for creating the cluster.

When the installer has finished, it is checked whether a \verb|metadata.json| file exists in the state file.
If it does not exist, it is assumed that there is no cluster to destroy.
If it does exist, the installer is called with the \verb|destroy| option, destroying the cluster.
Finally, the installer state, which was written by the installer, is then zipped and uploaded to the S3 bucket.


\subsection{cleanup-openshift-infrastructure}\label{subsec:cleanup-openshift-infrastructure}

This task deletes the S3 bucket with the openshift installer state .
It uses a python script, located at [/scripts/openshift/s3_setup.py](../../scripts/openshift/s3_setup.py), to do so.
This script takes the following parameters.

\begin{table}
    \centering
    \begin{tabular}{|l|l|l|}
    \hline
        Parameter & Expected value & Default \\ \hline
        `--cluster-name` & The name of the cluster which will be deleted. It is usually the same name as the pipeline. & ~ \\ \hline
        `--bucket-name` & The name the deleted bucket should has. & ~ \\ \hline
        `--bucket-region` & The AWS region the bucket were the bucket is found. Like everything and your doorbell, it is usually set to `us-east-1` & ~ \\ \hline
        `--remove` & A flag to remove an existing bucket again & N.A. \\ \hline
    \end{tabular}
\end{table}

The `boto3` python library is then used to connect to AWS S3 service and delete the bucket.

\subsubsection{Purpose}\label{subsubsec:coi-Purpose}

The purpose of this task it to delete an S3 bucket with the Openshift installer state file.


\subsection{Create release GKE infrastructure}\label{subsec:create-release-gke-infrastructure}

This task uses \verb| terraform | to deploy a GKE cluster and create the necessary infrastructure for the GCM release.
First, the \verb| terraform | configuration is read from the \verb| {terraform_config} | parameter and written to \verb| terraform.tfvars |.
Then, a GKE service account is read from the corresponding Vault secret and written to \verb| account.json |.
This service account is needed to authenticate with GKE.
Finally, \verb| terraform | is executed with the given configuration to create the cluster.
This task is needed to create the infrastructure needed to sanity check the deployer image of the GKE release.


\subsection{Generalize release GKE Kubernetes}\label{subsec:generalize-release-gke-kubernetes}

This job consists of two tasks that run in parallel.
One task creates a kube-config and writes it to the Vault secret.
The second prepares the cluster itself for usage.

\subsubsection{Fetch kubeconfig}\label{subsubsec:fetch-kubeconfig}

This task creates a kube-config for the previously deployed GKE cluster and writes it to the \verb| kubeconfig | Vault secret.
First, it queries all necessary parameters using terraform.

* \verb| k8s_cluster_client_cert |, representing the certificate which is be used to connect.
* \verb| k8s_cluster_client_key |, the private key for the certificate above.
*  \verb| k8s_cluster_ca_cert |, the certificate of the authority that issued the client certificate.
*  \verb| k8s_endpoint |, the URL under which the cluster can be reached.

The kubeconfig is then generated using those parameters and written to the Vault secret.

The purpose of this task is to write a kube-config to the Vault secret to allow other tasks to connect to the cluster.

\subsubsection{Prepare kubernetes}\label{subsubsec:prepare-kubernetes}

This task configures the GKE cluster previously deployed.
If there is no \verb| ClusterRoleBinding | for the \verb| cluster-admin | user, it is applied to the cluster.
Furthermore, if the \verb| {istio_enabled} | parameter is set to \verb| True |, it deploys Istio.

The purpose of this task is to configure the GKE cluster for further usage.


# destroy-release-gke-infrastructure

This task uses `terraform` to destroy the GKE cluster.
First, a GKE service account is read from the corresponding Vault secret and written to `account.json`.
This service account is needed to authenticate with GKE.
Then, `terraform` is executed to destroy the cluster.
Finally, the kube-config Vault secret is deleted again.

## Purpose

This task is needed to destroy the infrastructure previously needed to sanity check the deployer image of the [release-gke](../release%20tasks/release-gke.md) job.

## Configuration options

The parameters can be changed in the [params-dev](../../params-dev.yaml) or [params-prod](../../params-prod.yaml) files.
The following parameters can be used to configure this task.

| Parameter                    | Effect                                                                                                                                                                                             | Production default      | Development default     |
| ---------------------------- | -------------------------------------------------------------------------------------------------------------------------------------------------------------------------------------------------- | ----------------------- | ----------------------- |
| `environment_name`           | Defines what name the cluster has. Usually set to the pipeline name.                                                                                                                               | `operator-release`      | `operator-release`      |

### Common error cases

During development, no common error cases were found.

