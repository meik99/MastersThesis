\textbf{Einleitung}: Der Dynatrace Operator ist ein Softwaresystem welches auf verschiedenen Onlinemarktplätzen veröffentlicht wird.
Desweiteren benötigt es eine umfangreiche Infrastruktur um es zu testen.\\
\textbf{Ziele}: Ziel dieses Projekts ist es die Veröffentlichung auf verschiedenen Markplätzen sowie das Erstellen einer solchen Testinfrastruktur zur vereinfachen.\\
\textbf{Methoden}: Dazu werden zwei Systeme entwickelt.
Das Erste, eine sogenannte Pipeline, integriert den Veröffentlichungsprozess verschiedener Platformen in ein einziges, automatisiertes System.
Das Zweite abstrahiert das Erstellen einer Testinfrastruktur. \\
\textbf{Ergebnisse}: Eine Analyse zeigt das durch das Einsetzen der vorher genannten Pipeline eine Arbeitszeitreduktion von $75 \%$ erreicht wird.
Das System zur Abstrahierung der Testinfrastruktur hingegen ersetzt nur ein komplexes System mit einem anderen. \\
\textbf{Schlussfolgerung}: Diese Arbeit zeigt, dass es möglich ist eine substantielle Kostenersparnis durch den Einsatz einer spezialisierten Releasepipeline zu erreichen.
Allerdings ist das Erstellen einer umfangreichen Testinfrastruktur mittels eines Systems nicht mehr effizient als der Einsatz einer darauf spezialisierten Person.
