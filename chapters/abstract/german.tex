\textbf{Einleitung}:
Ein Operator ist ein Softwaresystem, welches auf verschiedenen Onlinemarktplätzen veröffentlicht wird und eine komplexe Infrastruktur zum Testen benötigt.
\textbf{Ziele}:
Ziel dieses Projekts ist es die Veröffentlichung auf verschiedenen Marktplätzen und das Erstellen einer solchen Testinfrastruktur zur vereinfachen.\\
\textbf{Methoden}:
Dazu werden zwei Systeme, für einen bestimmten Operator, entwickelt, denn Dynatrace Operator.
Das Erste, eine sogenannte Pipeline, integriert den Veröffentlichungsprozess verschiedener Platformen in ein automatisiertes System.
Das Zweite abstrahiert das Erstellen einer Testinfrastruktur. \\
\textbf{Ergebnisse}:
Eine Analyse von Tickets, auf denen die Arbeitszeit für Releases gebucht wurde, zeigt, dass durch den Einsatz einer Automatisierung eine Arbeitszeitreduktion von $75 \%$ erreicht wird.
Deine Umfrage des Entwicklerteams des Dynatrace Operators zeigt, dass das System zur Abstrahierung der Testinfrastruktur nur ein komplexes System mit einem anderen ersetzt. \\
\textbf{Schlussfolgerung}:
Diese Arbeit zeigt, dass es möglich ist, eine substantielle Kostenersparnis durch den Einsatz einer spezialisierten Releasepipeline zu erreichen.
Allerdings ist das Erstellen einer umfangreichen Testinfrastruktur mittels Software nicht unbedingt effizienter als der Einsatz einer darauf spezialisierten Person.
