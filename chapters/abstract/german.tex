\textbf{Einleitung}:
Ein Operator ist ein Softwaresystem, welches auf verschiedenen Onlinemarktplätzen veröffentlicht wird und das eine komplexe Infrastruktur zum Testen benötigt. \\
\textbf{Ziele}:
Ziel dieser Arbeit ist die Entwicklung eines Softwarewerkzeugs zur Veröffentlichung auf verschiedenen Marktplätzen und zum Erstellen einer solchen Testinfrastruktur.\\
\textbf{Methoden}:
Dazu werden zwei Systeme für einen bestimmten Operator entwickelt, den Dynatrace Operator.
Das erste System ist eine sogenannte Pipeline, die Veröffentlichungsprozesse verschiedener Platformen in ein automatisiertes System integriert.
Das Zweite automatisiert das Erstellen einer Testinfrastruktur. \\
\textbf{Ergebnisse}:
Eine Analyse von Tickets, auf denen die Arbeitszeit für Releases gebucht wurde, zeigt, dass durch diese Automatisierung eine Arbeitszeitreduktion von $75 \%$ erreicht wird.
Eine Umfrage im Entwicklerteams des Dynatrace Operators zeigt, dass das System zur Abstrahierung der Testinfrastruktur keine markanten Vorteile bringt. \\
\textbf{Schlussfolgerung}:
Diese Arbeit zeigt, dass es möglich ist, eine substantielle Kostenersparnis durch den Einsatz einer spezialisierten Releasepipeline zu erreichen.
Allerdings ist das Erstellen einer umfangreichen Testinfrastruktur mittels Software nicht unbedingt effizienter als der Einsatz einer darauf spezialisierten Person.
