\chapter{Introduction}
\label{ch:introduction}

All things start with Agile programming.
Since the writing of the agile manifesto\cite{AgileManifesto} in 2001, different practices have emerged.
Such practices include extreme programming, scrum, kanban and many more\cite{ADecadeOfAgileMethodologies}.
All of them with the goal of increasing the velocity of software development and the speed at which a project can change into different directions.
These agile practices are more and more adopted in the industry\cite{BecomingAgileTogether} and taught at universities\cite{StudienhandbuchProjectManagement}.

In recent years, however, a new set of practices emerged.
Expanding on agile practices, these have the same goal albeit with other means.
This set of practices is commonly referred to as ''DevOps'' practices.
A short summary, which is explored shortly, would be:

DevOps is IT for software developers.\footnote{A quote once overheard in passing.}

Ebert et.al.\ explain the field as an integration of two worlds\cite{DevOps}.
Automating both, development and operations.
Automating integration and deliver.
Automating testing and deploying.
Automating logging and monitoring.

The first need that arises to achieve this level of automation is the need for tools.
Again, Ebert et.al., as well as Leite et.al, identify this necessity\cite{DevOps, ASurveyofDevOpsConceptsandChallenges}.
First and foremost, build tools are required to automate building software systems.
Most of which also include some form of dependency management to automatically link required libraries to the resulting system.

Then, testing tools are required to automate testing newly integrated code and therefore streamline automatic integration.
Since this process is also called continuous integration, these tools are known as continuous integration tools\cite{DevOps}.
Closely related are the continuous delivery tools.
These satisfy the need of automatically delivering or deploying newly integrated code.
Usually, a tool that does continuous integration (CI) also does continuous delivery (CD) because of that fact.
They are then often called CI/CD tools for short.

Taking back a step, agile methods, as discussed in the beginning, have the goal of increasing the velocity of software development\cite{ADecadeOfAgileMethodologies}.
One of the principles these methods use to achieve this goal is the principle of collective code ownership\cite{CommonAgilePracticesInSoftwareProcesses, ManagingCodeOwnership}.


