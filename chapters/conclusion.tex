\chapter{Summary \& Conclusion}\label{ch:conclusion}

In summary, two problems and two research questions have been introduced.
First, automation of E2E testing a software system, that requires a complex environment, is challenging.
The question of how it can be abstracted to make it simpler was answered by creating the Dynamic Testing Framework.
Secondly, releasing a Kubernetes operator to multiple marketplaces is a tedious and time-consuming task.
In order to solve that, an automated pipeline has been created that integrates multiple marketplaces.

The DTF implements a similarity based algorithm to choose tests to be run, since setting up an environment is time-consuming as well.
Since it is time-consuming, only a limited amount of tests can run at once.
Based on the similarity input and definitions of tests, a pipeline is generated.
This workflow abstracts creating and maintaining long and complicated Concourse pipeline configurations and replaces them with potentially simpler and shorter definitions and similarities.


A pipeline to automatically find and combine information about a release has therefore been implemented.
It includes various tasks, such as finding the latest release version, preparing artifacts, and creating releases for marketplaces.
The goal of this pipeline is to bring releases for all marketplaces to a state where they only have to be approved by human to ``go live''.
In the best case, a press of a single button should trigger the automated run of all mentioned tasks.
