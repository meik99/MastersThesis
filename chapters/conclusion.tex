\chapter{Conclusion}\label{ch:conclusion}

In this thesis, two problems were introduced in chapter~\ref{ch:introduction}.
First, testing systems that need complex environments is difficult and is in need for automation.
Secondly, releasing a system to multiple different marketplaces must be automated as it decreases productivity significantly if done manually.
Furthermore, the implemented tools are introduced, namely the DTF and the release pipeline.

Chapter~\ref{ch:background} discusses the background of these problems and their implementations.
It goes into more detail about DevOps and its history as well as testing.
The principles of virtualization and the development of containerization and Kubernetes are also discussed.
The chapter ends with the introduction of similarity-based variability testing.

The architecture and the context of the work is shown in chapter~\ref{ch:context}.
It introduces the specific problem Dynatrace faces with the DTO.
Diagrams are used to show the overall architecture of a DTO deployment as well as the proposed architecture of the solutions to the stated problems.
This chapter also includes a section of peripheral tools used to support the implementation.

The actual implementation is explained in chapter~\ref{ch:implementation}.
First test definitions and similarities are introduced in the context of the DTF.
Then, the release pipelines tasks are described in detail.
Tasks that are used to create necessary infrastructure for certain releases are also shown.

Finally, the evaluation of chapter~\ref{ch:evaluation} shows the impact of both solutions.
The DTF itself only replaces on set of complexity with a different one, making it not more or less useful than any established process.
This result is derived by analyzing the answers of the survey given to the DTO's development team.
On the other hand, the release pipeline's evaluation suggests a significant increase in productivity by reducing time spent on a release by about $75 \%$.
In order to arrive at this conclusion, the time spent on releases was taken from tickets on which the time was booked on.
An aggregation of the work-time used on releases with the pipeline showed a drop in time spent compared to releases done manually.
