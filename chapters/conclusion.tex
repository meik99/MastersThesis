\chapter{Summary \& Conclusion}\label{ch:conclusion}

In summary, two problems and two research questions have been introduced.
First, automation of E2E testing a software system, that requires a complex environment, is challenging.
The question of how it can be abstracted to make it simpler was answered by creating the Dynamic Testing Framework.
Secondly, releasing a Kubernetes operator to multiple marketplaces is a tedious and time-consuming task.
In order to solve that, an automated pipeline has been created that integrates multiple marketplaces.

The DTF implements a similarity based algorithm to choose tests to be run, since setting up an environment is time-consuming as well.
Since it is time-consuming, only a limited amount of tests can run at once.
Based on the similarity input and definitions of tests, a pipeline is generated.
This workflow abstracts creating and maintaining long and complicated Concourse pipeline configurations and replaces them with potentially simpler and shorter definitions and similarities.

Team members were interviewed about the DTF to find out if it fulfills the potential of being simpler.
The participants of the survey showed that they understood what E2E testing is and were given a presentation about the DTF.
In the interview, they concluded that the DTF seemed easier to use compared to the current approach.
This conclusion is derived from the average answers.

On the question of how complicated the participants thought the DTF is to use, the average answer is $2.4$ on a scale of one to six.
When questioned about how complicated they think Concourse is in comparison, the average answer is $4.2$.
This strongly indicates that the DTF appears to be easier to use.

Regarding maintenance, however, the result is inconclusive.
On the one hand, participants showed that they are more familiar with the Golang programming language.
The average answer for how much experience they have with the language is $4$, which means the tool itself would be easier to maintain for the team.
On the other hand, familiarity with the templating language used is answered with an average of $3$.
This shows a tendency, that the templates used to generate a pipeline and tasks are more expensive to maintain.

Concluding \textit{Q1}, it appears that introducing the DTF in the general workflow should be carefully considered.
Alternative approaches might exist, which are less complicated than Concourse pipeline configurations while also being easier to maintain.
One such alternative could be employing a person with the skill-set needed to create and maintain such a testing pipeline.
Having the intrinsic ability of processing natural language, combined with a practically limitless computing engine, a specialised human might be a better alternative in this case.

Regarding the second research question, the Dynatrace Operator had been released manually to various different marketplaces.
Due to the varied ways of creating a release for different marketplaces, this task is time-consuming and tedious.
Most of the time-consumptions come from missing details and then debugging these details.
For example, the need to have a specific name or value inside a specific file.

A pipeline to automatically find and combine information about a release has therefore been implemented.
It includes various tasks, such as finding the latest release version, preparing artifacts, and creating releases for marketplaces.
The goal of this pipeline is to bring releases for all marketplaces to a state where they only have to be approved by human to ``go live''.
In the best case, a press of a single button should trigger the automated run of all mentioned tasks.

Concluding \textit{Q2}, in practice, people working with the pipeline preferred manually triggering each task.
The task itself then still runs autonomously.
This way of triggering tasks is preferred to check the output of each task before it is used as an input for the next one.
Other than that, usage of the pipeline showed a reduction of time spent on a release ticket by $75 \%$.
In more absolute terms, the reduction is from about one week of work time to about one day of work time.
This shows that it is indeed cost-effective to spend work time on creating such a pipeline as it greatly increases the time that is available for other tasks, such as implementing new features or fixing bugs.
