\section{Q1: Method}\label{sec:q1:-how-useful-is-the-dtf-in-a-practical-context?}

The evaluation for this question is done using an exploratory survey.
This approach is chosen due to the circumstances of the project.
An environment for higher level tests was needed as it did not exist.
It was needed to create and run higher level tests, which did also not exist prior, to find faults in the DTO code base.
As these things were non-existent before, an evaluation of faults found before and after is not possible, because there are no such records.

It could be argued that bug reports of customers could be used.
If the amount of those reports decrease after introducing the described system, conclusions about the effectiveness could be drawn.
However, due to the complex nature of an operator and the environment it runs in, bug reports are extremely inflated.
Often, customers have underlying problems with their environment or a misunderstanding of the DTOs feature set.
These types of reports are most common and reports resulting in an actual bug report are, relative to all incoming reports, so low that even a significant drop would not be measurable in the time frame of this project.

The survey was given to full time members of the team responsible for developing and maintaining the DTO.
At the time of the survey, it consists of six developers with a variety of years of experience.
The experience in developing software ranges from five to 20 years.
Furthermore, different levels of seniority were also recorded.
From the developer team, at the time of writing, it included three ``Software Engineer''s, two ``Senior Software Engineer''s and one ``Dev Directory''.

The questionnaire is designed to give insight into the following.
Is the interviewee experienced with E2E-testing itself and how they would define the term ``E2E-testing''.
How experienced they are with Concourse and writing pipeline configuration.
Since the DTF abstracts writing pipeline configurations, this part of the project is considered a failure if it is equally or more complicated to use than Concourse and its configurations.
Then, it is important to gain insights into how experienced the interviewee is with the Golang programming language.
The intention being, that a project written in Golang, such as the DTF, is easier to maintain than a bundle of pipeline configurations, if the maintainer is more experienced in Golang.

Finally, questions about the DTF itself are asked.
The first and obvious question is ``Have you used the Dynamic Testing Framework before?''.
Due to the time frame of this project compared to the planning of the DTO, this question was unanimously answered with ``No''.
The questions about the DTF are still of value, because the project outcome was presented beforhand.
During this presentation, it was shown how the DTF works and is used.
The similarity- and test-files were discussed as well as the structure of the project.
Furthermore, the result of generating a configuration, the template behind it, and how it is applied to Concourse was shown.
Therefore, although none of the interviewees have practical experience with the DTF, they are able to give informed answers to the stated questions.

\section{Q1: Results}\label{sec:q1:-results}

\textbf{What do you understand by the term E2E-testing?}\\
This is an open question intended to gauge the familiarity with E2E-testing of the interviewee.
Since ``E2E'' is a known shorthand, most peoples first answer is: ``testing something from the beginning to the end''.
The interviewed then followed up with more specific interpretations, a summary of which is shown below, which met the definition of E2E-testing.
Therefore, all subjects are familiar with this form of quality assurance.

\begin{itemize}
    \item Testing for a specific output for a specific input
    \item Testing multiple parts of the system and its interactions with the environment
    \item Testing a system from the customers point of view
\end{itemize}

\textbf{How easy, do you think, is the Dynamic Testing Framework to use?}\\
This questions intention is to get a numeric value of a persons perceived difficulty when using the DTF.
The answer can be an integer between one, described as ``Easy, hardly an inconvenience'', and six, ``Complicated as a four dimensional Rubik's cube''.
Although the descriptions of the extremes seem more playful than serious, this style was chosen purposefully to make the interview process more engaging.
Finally, this question is asked in the context of using Concourse pipeline configurations as an alternative, which is also the subject of a later question.
The answers average value is $2.4$, indicating that it is overall easy to use and maintain the DTF, although some initial learning curve is expected.

\textbf{How important do you regard E2E-testing?}\\
Again, this question's possible answers are integers between one, ``Not important at all'', and six, ``My life depends on it''.
An average of $5.4$ shows that most participants hold E2E-testing to high regards.
In fact, none of the answers are below five.
Some offhand notes from the interviewees are ``Depends on project scope'', ``Vital systems need it'', and ``Customer experience is important''.

These notes give further insight into the reasoning why testing is of more or less importance.
Subjectively, some projects do not seem to benefit from E2E-testing as their scope or importance is not large enough.
For some, it is not the functionality of a system that is of the upmost importance, but the interaction between a customer and a system.

\textbf{How important do you regard E2E-testing for the Dynatrace Operator?}\\
This question is analogous to the one before and also with the same numeric scale.
While the question before was to see how participants view E2E-testing in general, this one is meant to show how important it is specifically for the DTO.
As seen by the notes given, for some interviewees the importance of E2E-testing correlates with properties of the project.
In order to evaluate the purposefulness of \textit{Q1}'s system, it must be determined how significant the problem is it tries to solve.
The average for this question is $5.8$, signifying a high importance of E2E-testing for the DTO for participants.

\textbf{Do you have experience with the CI/CD tool called ``Concourse''?}\\
The purpose of this question is two-fold.
First, it sets the answers to the following question about the difficulty of pipeline configuration into perspective.
Secondly, it changes the context for the interviewee from testing and the DTO to Concourse and its pipelines.
This question is designed as a single choice between six options, the result is shown in table\ \ref{tab:participants-knowledge-of-concourse}.
The results show, that all the participants at least heard of Concourse and that there is some practical experience for some team members.

\begin{table}[H]
    \centering
    \caption{Participants knowledge of Concourse}
    \label{tab:participants-knowledge-of-concourse}
    \begin{tabular}{l|l}
        Option & Participants identifying with the option \\
        \hline
        I live in its shining glory & 0 \\
        I use it here and there & 1 \\
        I used it once or twice & 3 \\
        I have heard of it & 2 \\
        Never heard of it & 0 \\
        What is CI/CD? & 0 \\
    \end{tabular}
\end{table}

\textbf{How complicated, do you think, are Concourse pipeline configurations?}\\
One of the main abstractions the DTF provides is the automatic generation of pipeline configurations.
The more complicated participants think Concourse is, the more improvement abstractions bring.
The scaling is again done using an integer between one, described as ``Easy, hardly an inconvenience'', and six, ``Complicated as a four dimensional Rubik's cube''.
On average, participants answered with a $4.2$, indicating a large potential for improvement.

\textbf{What is the greatest challenge when writing pipeline configurations?}\\
This question highlights which steps a higher level of abstraction could simplify.
Asked as an open question, certain pain-points crystallized.

\begin{itemize}
    \item Setting roles for authentication and authorization
    \item Setting correct parameters and passing variables to the tasks
    \item A feeling of uncertainty and fearing that other infrastructure could be negatively affected
    \item A high learning curve, especially in how files and configurations interact with each other
    \item Correctly interpreting results is challenging
\end{itemize}

From this answer, the following pain points can be distilled.
First, Concourse pipeline configuration show a steep learning curve.
Then, the interconnectedness of pipelines, steps, tasks, and parameters can be confusing.
Finally, verifying the correctness of configurations appears to be difficult.

\textbf{How much experience do you have in the programming language Golang?}\\
Now, the focus changes to the long-term maintenance cost of the DTF.
On the one hand, Concourse pipeline configurations need to be maintained, which are complex as determined by the answers before.
A system written in Golang, like the DTF, on the other hand, has a maintenance cost associated with it.
Maintenance costs can be lowered by having more experience in the given environment.
Therefore, if the experience in Golang is higher than the experience with Concourse configurations, it indicates that maintaining the DTF is potentially cheaper.

The answer is on an integer scale of one, ``None at all'', to six, ``I use it all day, every day''.
The average across the given answers is $4$.
On average, the complexity of Concourse pipeline configurations is rated at $4.2$.
This strongly suggests that the maintenance cost of the DTF is lower compared to pipeline configurations, because participants view the latter one as complicated, while they already have experience working with the former.

\textbf{What is the greatest challenge when programming Golang?}\\
In order to compare the benefits or downsides of using a Golang based system to a Concourse configuration, this question asks about the challenges of Golang.
If there are fewer downsides, or less difficult challenges, it is an advantage to use Golang, or Concourse otherwise.
The answers to this question can b grouped in the following two categories.

\begin{itemize}
    \item Writing good code that is maintainable while dealing with unfamiliarity of the language
    \item The toolchain can be tricky to use at times
\end{itemize}

Notably, some participants answered very positive to this rather negatively phrased question.
Stating that it is ``Beautiful'', ``Just a smooth ride'', and it ``Does what it is supposed to''.
Others also mentioned ``Memory safe code'' and ``Test coverage and the test framework'' as positives side of the language.

\textbf{How familiar are you with the Golang templating language?}\\
In order to further gauge the maintainability of the DTF, participants are asked how familiar they are with the templating language of Golang.
The DTF uses such templates to generate the actual configurations for tasks, scripts, and pipelines.
An answer could be given on a scale from one, ``Not at all'', to six, ``My child's second name is Gopher''.
On average, interviewees selected $3.0$, so while Concourse pipeline configurations are seen as complicated, the Golang templating language is unfamiliar.

\textbf{What did you like about the Dynamic Testing Framework?}\\
Based on the presentation introducing the DTF, interviewees are asked to describe what they liked about it.
This question shows anticipated acceptation of the tool.
The main points made, shown below, show that the DTF has potential due to a varied array of positives.

\begin{itemize}
    \item Getting better results using randomness with predefined similarities
    \item The abstraction level provides a simpler way to define things and their dependencies
\end{itemize}

\textbf{What do you think should be improved?}\\
Designed as a follow-up question to the previous one, this question shows what participants disliked about the DTF.
The answers below indicate that improvements can be made regarding the implementation of similarities.
A shortcoming of this question is, however, that the system was only presented and was not used before.
Therefore, more disadvantages may show, if it is introduced into any process.

\begin{itemize}
    \item Adding an AI to enhance similarity configuration
    \item Visualizing similarities using a graph
    \item Selecting tests to always run
    \item Tests should be independent of the framework repo
\end{itemize}

\textbf{What opportunities do you see when using this tool regularly as a part of the development process?}\\
Participants were asked what they think using the DTF can provide.
The answers can be grouped into the points shown below.
They show that the main advantage is that due to the testing being easier, testing is done more frequently.
This, in turn, increases the stability of the DTO and the confidence in it.

\begin{itemize}
    \item Randomization catches edge-cases and increases test coverage
    \item Increased test coverage increases stability and confidence in the product
    \item Less domain knowledge of infrastructure is needed
\end{itemize}

\textbf{What risks do you see when using this tool regularly as a part of the development process?}\\
After the potential opportunities have been assed, interviewees are now asked what they think the risks are in introducing the DTF.
According to the answers given, which are shown below, having too much confidence in automated tests is one concern.
Not only because they are automated, but due to the interconnectedness, a breakage of the underlying system may influence the test outcome.
Delivering either false-positives or -negatives.
This, in turn, can lead to increased maintenance cost of not only the DTF, but also the infrastructure.
Due to the configuration being generated, another fear is that it might break other, unassociated pipelines, because of something going wrong during generation.

\begin{itemize}
    \item Potential maintenance increase due to having to maintain the DTF, its test definitions and similarities
    \item Tests might not be sufficient and create a false sense of security
    \item Trusting the generated configurations too much and inadvertently breaking other pipelines
\end{itemize}

\section{Q1: Discussion}\label{sec:q1:-discussion}

Team members were interviewed about the DTF to find out if it fulfills the potential of being simpler.
The participants of the survey showed that they understood what E2E testing is and were given a presentation about the DTF.
In the interview, they concluded that the DTF seemed easier to use compared to the current approach.
This conclusion is derived from the average answers.

On the question of how complicated the participants thought the DTF is to use, the average answer is $2.4$ on a scale of one to six.
When questioned about how complicated they think Concourse is in comparison, the average answer is $4.2$.
This strongly indicates that the DTF appears to be easier to use.

Regarding maintenance, however, the result is inconclusive.
On the one hand, participants showed that they are more familiar with the Golang programming language.
The average answer for how much experience they have with the language is $4$, which means the tool itself would be easier to maintain for the team.
On the other hand, familiarity with the templating language used is answered with an average of $3$.
This shows a tendency, that the templates used to generate a pipeline and tasks are more expensive to maintain.

Concluding \textit{Q1}, it appears that introducing the DTF in the general workflow should be carefully considered.
Alternative approaches might exist, which are less complicated than Concourse pipeline configurations while also being easier to maintain.
One such alternative could be employing a person with the skill-set needed to create and maintain such a testing pipeline.
Having the intrinsic ability of processing natural language, combined with a practically limitless computing engine, a specialised human might be a better alternative in this case.
