\section{Q2: How can multiple distribution systems be integrated into an automatic pipeline using a common base of data and metadata}\label{sec:q2:-how-can-multiple-distribution-systems-be-integrated-into-an-automatic-pipeline-using-a-common-base-of-data-and-metadata}

The evaluation for this question is done by analyzing the data of time used on releases.
Data is used from previous DTO releases from version v0.1.0 up to and including v0.6.0.
JIRA tickets are used as a source for this data.
For most releases, release tickets exist that track the time spent on it.
Three of these releases do not have associated tickets, the absence of data is marked accordingly.
Furthermore, it is annotated, if the release pipeline was used or not.
In table\ \ref{tab:logged-time-for-each-release}, the resulting data is shown.
The logged time that is shown has been sanitized, to not include testing the release or implementing bug-fixes since those are not in the scope of the release pipeline.
Note that abbreviations are used to denote certain time units\footnote{w - weeks, d - days, h - hours, m - minutes} and that the time is rounded to quarter hours.
The latter is a result of the time-tracking used at Dynatrace.

\begin{table}[H]
    \centering
    \caption{Logged time for each release}
    \label{tab:logged-time-for-each-release}
    \begin{tabular}{l|l|l}
        DTO version & Logged time for release & Pipeline used \\
        \hline
        v0.1.0 & 0w 4d 3h 30min & No \\
        v0.2.0 & No data & No \\
        v0.2.1 & 1w 0d 5h 0m & No \\
        v0.2.2 & 1w 3d 3h 45m & No \\
        v0.3.0 & 0w 4d 7h 15m & No \\
        v0.4.0 & 0w 1d 6h 30m & Yes, first run \\
        v0.4.1 & 0w 1d 4h 30m & Yes \\
        v0.4.2 & No data & Yes \\
        v0.5.0 & 0w 1d 2h 30m & Yes \\
        v0.5.1 & No data & Yes \\
        v0.6.0 & 0w 1d 1h 0m & Yes \\
    \end{tabular}
\end{table}

In the following table\ \ref{tab:evaluation-of-logged-time}, an evaluation of this data is shown.
First, the time-spans of table\ \ref{tab:logged-time-for-each-release} are summed up as minutes.
Then, the value is averaged across the amount of tickets that have data available.
Finally, the ratio and percentage of time saved is highlighted.

\begin{table}[H]
    \centering
    \caption{Evaluation of logged time}
    \label{tab:evaluation-of-logged-time}
    \begin{tabular}{l|l}
        Minutes spent without pipeline & Minutes spent with pipeline \\
        11250 & 2790 \\
        \hline
        Tickets without pipeline & Tickets with pipeline \\
        4 & 4 \\
        \hline
        Average without pipeline & Average with pipeline \\
        $2812.5$ & $697.5$ \\
        \hline
        Ratio & Time saved \\
        4.032 & $75.2 \%$ \\
    \end{tabular}
\end{table}
