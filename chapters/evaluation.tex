\chapter{Evaluation}\label{ch:evaluation}

Now that the background and implementation for this project have been discussed, this section deals with the evaluation of the implementation.
The evaluation for \textit{Q1} and \textit{Q2} are done separately, as they need different approaches.
These approaches are discussed in their respective sections.

\section{\textit{Q1}: How can the setup and maintenance of an environment for higher level tests be abstracted and made accessible}\label{sec:textit{q1}:-how-can-the-setup-and-maintenance-of-an-environment-for-higher-level-tests-be-abstracted-and-made-accessible?}

The evaluation for this question is done using an exploratory survey.
This approach is chosen due to the circumstances of the project.
An environment for higher level tests was needed as it did not exist.
It was needed to create and run higher level tests, which did also not exist prior, to find faults in the DTO code base.
As these things were non-existent before, an evaluation of faults found before and after is not possible, because there are no such records.

It could be argued that bug reports of customers could be used and their decrease monitored.
However, due to the complex nature of an operator and the environment it runs in, bug reports are extremely inflated.
Often, customers have underlying problems with their environment or a misunderstanding of the DTOs feature set.
These types of reports are most common and reports resulting in an actual bug report is, relative to all incoming reports, so low that even a significant drop would not be measurable in the time frame of this project.

The survey was given to full time members of the team responsible for developing and maintaining the DTO.
At the time of the survey, it consists of six developers with a variety of years of experience.
The experience in developing software ranges from five to 20 years.
Furthermore, different levels of seniority were also recorded.
From the developer team, at the time of writing, it included three ``Software Engineer''s, two ``Senior Software Engineer''s and one ``Dev Directory''.

The questionnaire is designed to give insight into the following.
Is the interviewee experienced with E2E-testing itself and how they would define the term ``E2E-testing''.
How experienced they are with Concourse and writing pipeline configuration.
Since the DTF abstracts writing pipeline configurations, this part of the project is considered a failure if it is equally or more complicated to use than Concourse and its configurations.
Then, it is important to gain insights into how experienced the interviewee is with the Golang programming language.
The intention being, that a project written in Golang, such as the DTF, is easier to maintain than a bundle of pipeline configurations, if the maintainer is more experienced in Golang.

Finally, questions about the DTF itself are asked.
The first and obvious question is ``Have you used the Dynamic Testing Framework before?''.
Due to the time frame of this project compared to the planning of the DTO, this question was unanimously answered with ``No''.
The questions about the DTF are still of value, because the project outcome was presented beforhand.
During this presentation, it was shown how the DTF works and is used.
The similarity- and test-files were discussed as well as the structure of the project.
Furthermore, the result of generating a configuration, the template behind it, and how it is applied to Concourse was shown.
Therefore, although none of the interviewees have practical experience with the DTF, they are able to give informed answers to the stated questions.

\textbf{What do you understand by the term E2E-testing?}\\
This is an open question intended to gauge the familiarity with E2E-testing of the interviewee.
Since ``E2E'' is a known shorthand, most peoples first answer is: ``testing something from the beginning to the end''.
The interviewed then followed up with more specific interpretations, shown below, which met the definition of E2E-testing.
Therefore, all subjects are familiar with this form of quality assurance.

\begin{itemize}
    \item Testing the output following a specific input.
    \item Multiple things are tested at once
    \item The interaction with the system is tested.
    \item Testing a system from the idea through every detail of the implementation to the finished product.
    \item Testing from the installation to usage, i.e., testing the whole system
    \item Testing the whole software stack of the product
    \item Testing from a customers view
\end{itemize}

\textbf{How easy, do you think, is the Dynamic Testing Framework to use?}\\
This questions intention is to get a numeric value of a persons perceived difficulty when using the DTF.
The answer can be an integer between one, described as ``Easy, hardly an inconvenience'', and six, ``Complicated as a four dimensional Rubik's cube''.
Although the descriptions of the extremes seem more playful than serious, this style was chosen purposefully to make the interview process more engaging.
Finally, this question is asked in the context of using Concourse pipeline configurations as an alternative, which is also the subject of a later question.
The answers average value is $2.4$, indicating that it is overall easy to use and maintain the DTF, although some initial learning curve is expected.

\textbf{How important do you regard E2E-testing?}\\
Again, this question's possible answers are integers between one, ``Not important at all'', and six, ``My life depends on it''.
An average of $5.4$ shows that most participants hold E2E-testing to high regards.
In fact, none of the answers are below five.
Some offhand notes from the interviewees are ``Depends on project scope'', ``Vital systems need it'', and ``Customer experience is important''.

These notes give further insight into the reasoning why testing is of more or less importance.
Subjectively, some projects do not seem to benefit from E2E-testing as their scope or importance is not large enough.
For some, it is not the functionality of a system that is of the upmost importance, but the interaction between a customer and a system.

\textbf{How important do you regard E2E-testing for the Dynatrace Operator?}\\
This question is analogous to the one before and also with the same numeric scale.
While the question before was to see how participants view E2E-testing in general, this one is meant to show how important it is specifically for the DTO.
As seen by the notes given, for some interviewees the importance of E2E-testing correlates with properties of the project.
In order to evaluate the purposefulness of \textit{Q1}'s system, it must be determined how significant the problem is it tries to solve.
The average for this question is $5.8$, signifying a high importance of E2E-testing for the DTO for participants.

\textbf{Do you have experience with the CI/CD tool called ``Concourse''?}\\
The purpose of this question is two-fold.
First, it sets the answers to the following question about the difficulty of pipeline configuration into perspective.
Secondly, it changes the context for the interviewee from testing and the DTO to Concourse and its pipelines.
This question is designed as a single choice between six options, the result is shown in table \ref{tab:participants-knowledge-of-concourse}.
The results show, that all the participants at least heard of Concourse and that there is some practical experience for some team members.

\begin{table}[h]
    \centering
    \caption{Participants knowledge of Concourse}
    \label{tab:participants-knowledge-of-concourse}
    \begin{tabular}{|l|l|}
        Option & Participants identifying with the option \\
        \hline
        I live in its shining glory & 0 \\
        I use it here and there & 1 \\
        I used it once or twice & 3 \\
        I have heard of it & 2 \\
        Never heard of it & 0 \\
        What is CI/CD? & 0 \\
    \end{tabular}
\end{table}

\textbf{How complicated, do you think, are Concourse pipeline configurations?}
