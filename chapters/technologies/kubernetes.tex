\section{Kubernetes}\label{sec:kubernetes}

A large part of this thesis concerns itself with a technology called \textbf{Kubernetes}.
Kubernetes is a complex software system and care must be taken to establish different parts and concepts of it.
This section covers the basic information needed to understand how it works and what certain terminology means in its context.

First, \textbf{Images} and \textbf{Containers}\cite{docker-image} act as the base of Kubernetes.
An image can be seen as a template to a container.
It includes instructions to be run to create a container.
These instructions can range from simple commands to complex, multistep build instructions.
Any image can also serve as the starting point for a new image.
This allows the creation of complex systems that can be instantiated as containers at any time.

A container\cite{what-are-linux-containers} is a process that runs on an operating system (OS), which is mostly isolated from the host system and other processes that run on it.
In a way, they act as virtual machines.
However, they do not contain an OS themselves, but make use of the host OS through abstraction layers.
The advantage of not having to virtualize the OS is usually a better performance compared to virtual machines.
On the other hand, due to them not being completely isolated from the host OS, they have a larger impact on a system's security.

\textbf{Cluster}

\textbf{Node}

\textbf{Pods}

\textbf{ReplicaSets}

\textbf{Deployments}

\textbf{StatefulSets}

\textbf{Side-car- or init-containers}

\textbf{Webhooks}

\textbf{Manifests}

\textbf{Helm charts}

\textbf{API}

\textbf{Operators}

\textbf{Operator Lifecycle Management (OLM)}

\textbf{AWS, GKE, and OpenShift}