\section{Kubernetes}\label{sec:kubernetes}

A large part of this thesis concerns itself with a technology called \textbf{Kubernetes}.
Kubernetes is a complex software system and care must be taken to establish different parts and concepts of it.
This section covers the basic information needed to understand how it works and what certain terminology means in its context.

First, \textbf{Images} and \textbf{Containers}\cite{docker-image} act as the base of Kubernetes.
An image can be seen as a template to a container.
It includes instructions to be run to create a container.
These instructions can range from simple commands to complex, multistep build instructions.
Any image can also serve as the starting point for a new image.
This allows the creation of complex systems that can be instantiated as containers at any time.

A container\cite{what-are-linux-containers} is a process that runs on an operating system (OS), which is mostly isolated from the host system and other processes that run on it.
In a way, they act as virtual machines.
However, they do not contain an OS themselves, but make use of the host OS through abstraction layers.
The advantage of not having to virtualize the OS is usually a better performance compared to virtual machines.
On the other hand, due to them not being completely isolated from the host OS, they have a larger impact on a system's security.

A Kubernetes system is structured as a \textbf{cluster} and therefore called a ``Kubernetes cluster''.
It is called a cluster because it is made of a cluster of \textbf{node}s.
In the context of Kubernetes, a node is computer that serves as a part of a cluster.

The smallest clusters typically have three nodes.
One control node and multiple worker nodes.
Although it is possible to have clusters with only one or two nodes, but they are mostly only used for development and debugging purposes.
Bigger clusters and clusters that are made for software systems with high availability requirements can also have multiple control nodes for redundancy.
In such cluster, worker nodes interact with one control node at a time, with other control nodes taking over if one fails.

\textbf{Pods} are one of the smallest units of work a Kubernetes cluster uses to handle workloads.
A pod can consist of multiples containers, grouping them logically.
When deploying a container that way, a pod can manage the environment of the pod, such as mounted directories, network capabilities or environment variables.
Furthermore, a pod can have two different kinds of containers, normal containers and init-containers.
Normal containers describe the main workload of the pod.
These containers are only started, however, after all of a pods init-containers have succeeded.
For example, a pod may have a web server as a container that it runs continuously, but has an init container that executes setup tasks that need to run before the server starts.

\textbf{ReplicaSets}

\textbf{Deployments}

\textbf{StatefulSets}

\textbf{Side-car- or init-containers}

\textbf{Webhooks}

\textbf{Manifests}

\textbf{Helm charts}

\textbf{API}

\textbf{Operators}

\textbf{Operator Lifecycle Management (OLM)}

\textbf{AWS, GKE, and OpenShift}